\documentclass[a4paper]{article}
    \usepackage[margin=0in]{geometry}
    \usepackage{amssymb}
    \usepackage{multicol}

    \begin{document}
    \begin{multicols}{2}
        \scriptsize
        \noindent\underline{\textbf{Proof techniques}}: construction, contradiction, counterexample, case enumeration, induction, pigeonhole principle, proving cardinality, diagonalization\\
        \underline{\textbf{Languages, Strings}}\\
        \textbf{Alphabet}: finite set of symbols\\
        \textbf{String}: finite sequence of symbols from alphabet\\
        \textbf{Replication}: $w^0 = \varepsilon, w^{i+1} = w^i w$\\
        \textbf{Reverse}: $w^R = w = \varepsilon$ if $|w| = 0$, else $\exists a \in \Sigma$ and $\exists u \in \Sigma ^*$ such that $w = ua$, then define $w^R = {au}^R$\\
        \boldmath${(wx)}^R = x^r w^r$\unboldmath: induction on $|x|$, base case $|x| = 0$ so $x = \varepsilon$, consider any string x where $|x| = n+1$, then $x = ua$ for some character $a$ and $|u| = n$, so ${(wx)}^R = {(w(ua))}^R = {(wu)a}^R = a{(wu)}^R = a(u^R w^R) = (a u^R) w^R = {(ua)}^R w^R = x^R w^R$\\
        \textbf{Language}: set of strings (finite/infinite) from alphabet, uncountably infinite number of these (power set of $\Sigma ^*$)\\
        \boldmath$\Sigma ^*$\unboldmath: countably infinite with non-empty alphabet, enumerate with lexicographic order\\
        \boldmath$L_1 L_2$\unboldmath: $\{ w \in \Sigma ^* : \exists s \in L_1 (\exists t \in L_2 (w = st)) \}$\\
        \boldmath$L^*$\unboldmath: $\{ \varepsilon \} \cup \{ w \in \Sigma ^* : \exists k \geq 1 (\exists w_1, w_2, \ldots w_k \in L (w = w_1 w_2 \ldots w_k)) \}$ or $L^0 \cup L^1 \cup L^2 \cup \ldots$\\
        \boldmath$L^+$\unboldmath: $L L^*$ or $L^* - \{\varepsilon\}$ iff $\varepsilon \notin L$ or $L^0 \cup L^1 \cup L^2 \cup \ldots$\\
        \boldmath${(L_1 L_2)}^R = L_2^R L_1^R$\unboldmath: $\forall x (\forall y ({(xy)}^R = y^R x^R))$ from before, then ${(L_1 L_2)}^R = \{ {(xy)}^R : x \in L_1$ and $y \in L_2 \} = \{ y^R x^R : x \in L_1$ and $y \in L_2 \} = L_2^R L_1^R$\\
        \textbf{Decision problem}: problem to which answer is yes/no or true/false\\
        \textbf{Decision procedure}: answers decision problem\\
        \textbf{Machine power hierarchy}: FSM (regular), PDA (context-free), TM (semi-decidable \& decidable)\\
        \textbf{Rule of least power}: use least powerful language suitable for expressing info, constraints or programs on WWW\\
        \boldmath$firstchars(L)$\unboldmath: $\{ w : \exists y \in L (y = cx \wedge c \in \Sigma _L \wedge x \in \Sigma _L^* \wedge w \in c^*) \}$, closed under FIN but not INF (since result is first character $^*$)\\
        \boldmath$chop(L)$\unboldmath: $\{ w : \exists x \in L (x = x_1 c x_2, x_1 \in \Sigma_L^*, x_2 \in \Sigma_L^*, c \in \Sigma_L, |x_1| = |x_2|$, and $w = x_1 x_2)$\\, language where all strings have exact middle character removed, must have had odd length to begin with, closed under FIN but not INF (can get empty set if never odd length)\\
        \textbf{Extra}: to describe language with at least 2 different substrings of length 2 $L = \{ w \in {\{a,b\}}^* : \exists x, y (x \neq y \wedge |x| = 2 \wedge |y| = 2 \wedge Substr(x,w) \wedge Substr(y,w)) \}$\\
        \underline{\textbf{Finite State Machines}}\\
        \textbf{DFSM Quintuple}: $M = (K, \Sigma, \delta, s, A)$, $K$ = finite set of states, $\Sigma$ = alphabet, $\delta$ = transition function from $(K \times \Sigma)$ to $K$, $s \in K$ = initial state, $A \subseteq K$ = set of accepting states\\
        \textbf{Configuration}: element of $K \times \Sigma^*$, current state and remaining input\\
        \textbf{Yields relation}: $|-_M$, relates 2 configurations if $M$ can move from the first to the second in 1 step, $|-_M^*$ for 0 or more\\
        \textbf{Computation}: finite sequence of configurations for some $n \geq 0$ such that $C_0$ is an initial configuration, $C_n$ is of the form $(q, \varepsilon)$ for some state $q \in K_M$ and $C_0 |-_M C_1 |-_M C_2 |-_M \ldots |-_M C_n$\\
        \textbf{DFSM will halt in $|w|$ steps}: execute computation from $C_0$ to $C_n$, each step will consume one character, so $n = |w|$, $C_n$ is either accepting or rejecting configuration, so will halt after $|w|$ steps\\
        \textbf{Parity}: odd if number of 1 is odd for binary string\\
        \boldmath$MinDFSM(M: DFSM)$\unboldmath:\\
        Initialise classes with an accepting class \& non-accepting class\\
        For each class with more than 1 state\\
        For each state and character check which class it goes to\\
        If behaviour differs between states split them\\
        End for\\
        End for\\
        Go through all the classes again until no splitting happens\\
        Each class becomes its own state, transitions already defined above\\
        \textbf{Number of states $\geq$ equivalence classes in L}: suppose it is less than equivalence classes, then by pigeonhole principle there must be at least 1 state that contains strings from 2 equivalence classes, but then future behaviour on these two strings will be identical, which is not consistent with the fact that they are in different equivalence classes\\
        \textbf{NDFSM Quintuple}: replace $\delta$ with $\Delta$, transition relation, finite subset of $(K \times (\Sigma \cup \{\varepsilon\})) \times K$\\
        \textbf{NDFSM vs DFSM}: can enter configuration with input symbols left but no move available (halt without accepting), can enter configuration from which 2 or more competing transitions available ($\varepsilon$-transition, more than 1 transition for single input character)\\
        \boldmath$eps(q)$\unboldmath: $\{ p \in K : (q,w) |-^*_M (p,w) \}$, closure of $\{q\}$ under relation $\{ (p,r)$: there is a transition $(p, \varepsilon, r) \in \Delta \}$, to calculate initialise $result = \{q\}$, add all transitions $(p, \varepsilon, r) \in \Delta$ where $p \in result, r \notin result$ to $result$, then return $result$\\
        \boldmath$ndfsmtodfsm(M: NDFSM)$\unboldmath:\\
        Compute $eps(q)$ for each state $q$, $s' = eps(s)$ (initial state)\\
        Set $active\mbox{-}states = \{s'\}$ (set of set of states) and $\delta' = \varnothing$\\
        While $\exists Q \in active\mbox{-}states$ for which $\delta'$ has not been computed //computing $\delta'$\\
        For each $c \in \Sigma_M$\\
        Set $new\mbox{-}state = \varnothing$\\
        For each state $q \in Q$\\
        For each state $p : (q, c, p) \in \Delta$\\
        Set $new\mbox{-}state = new\mbox{-}state \cup eps(p)$\\
        End for\\
        Add $(Q, c, new\mbox{-}state)$ to $\delta'$, if $new\mbox{-}state \notin active\mbox{-}states$ insert it\\
        End for\\
        End for\\
        End while\\
        Set $K' = active\mbox{-}states$ and $A' = \{ A \in K' : Q \cap A \neq \varnothing \}$\\
        \textbf{Extra}: when making FSM may start with complement\\
        \underline{\textbf{Regular expressions}}\\
        \textbf{Allowable}: $\varnothing, \varepsilon$, every element of $\Sigma$, if $\alpha, \beta$ are regex then so are $\alpha\beta, \alpha\cup\beta, \alpha^*, \alpha^+, (\alpha)$, no actual need for rules for $\varepsilon$ and $\alpha^*$\\
        \textbf{Order of operations}: Kleene star, concatenation, union (high to low)\\
        \textbf{FSM \& Regex equivalence}: can create FSM to accept regex (do so for each rule), algorithm exists for other way\\
        \boldmath$fsmtoregexheuristic(M: FSM)$\unboldmath: remove unreachable states, if no accepting states return $\varnothing$, if start state part of loop create new start state $s$ \& connect $s$ to original start via $\varepsilon$-transition, if multiple accepting states create new accepting state \& connect old ones to new with $\varepsilon$-transition, if only 1 state return $\varepsilon$, rip out all states other than start \& accept, return regex\\
        \boldmath$fsmtoregex(M: FSM)$\unboldmath: $buildregex(standardize(M))$\\
        \boldmath$standardize(M: FSM)$\unboldmath: remove unreachable states, create start \& accepting if needed (from heuristic), if multiple transitions exist between 2 states collapse, create any missing transitions with $\varnothing$\\
        \boldmath$buildregex(M: FSM)$\unboldmath:\\
        If no accepting return $\varnothing$, if only 1 state return $\varepsilon$\\
        While states exist that are not start or accepting\\
        Select some state $rip$\\
        For every transition from $p$ to $q$\\
        If both $p$ \& $q$ are not $rip$ compute new transition ($R'(p,q) = R(p,q) \cup R(p,rip) {R(rip, rip)}^* R(rip, q$), either direct or via $rip$), then remove $rip$\\
        End for\\
        End while\\
        Return final regex\\
        \underline{\textbf{Regular grammar}}\\
        \textbf{Quadruple}: $(V, \Sigma, R, S)$, $V$ = rule alphabet, both nonterminals \& terminals, $\Sigma$ = terminals, $\subseteq V$, $R$ = finite set of rules, LHS single nonterminal, RHS is $\varepsilon$, single terminal or single terminal + single nonterminal, $S$ = nonterminal\\
        \textbf{Regular Languages \& Grammar equivalence}: construction, both ways\\
        \boldmath$grammartofsm(G)$\unboldmath: create separate state for each nonterminal, start state is $S$, if rule exist with single terminal RHS create new state labeled \#, for each rule $X \to aY$ create transition from $X$ to $Y$ labeled $a$, if $X \to a$ go to \#, if $X \to \varepsilon$ mark as accept, mark state \# as accept, if undefined (state, input) pairs remaining point them to dead state \& add loops to dead state for each character\\
        \underline{\textbf{Nonregular languages}}\\
        \textbf{Number of regular languages}: countably infinite, upper bound is number of FSM/regex, lower bound is every element of $a^+$ as its own language\\
        \textbf{Every finite language is regular}: union them all\\
        \textbf{Show that L is regular}: finite, FSM, regex, finite equivalence classes, regular grammar, closure theorems\\
        \textbf{Closure properties}: union, concatenation, Kleene star, complement (swap accept \& not, need all transitions explicitly, dead states \& DFSM), intersection ($\neg(\neg L_1 \cup \neg L_2)$, De Morgan's law), difference ($L_1 \cap \neg L_2$), reverse (turn start to accept, create new start connected by $\varepsilon$ to accepting states, flip transitions), letter substitution ($letsub(L_1) = \{ w \in \Sigma_2^* : \exists y \in L_1 \wedge w = y$ except every character $c$ of $y$ is replaced by $sub(c) \}$, $sub$ = function from $\Sigma_1$ to $\Sigma_2^*$)\\
        \textbf{Pumping theorem}: $\exists k \geq 1 (\forall $strings $w \in L, $where$ |w| \geq k (\exists x, y, z (w = xyz, |xy| \leq k, y \neq \varepsilon, \forall q \geq 0 (x y^q z$ is in L$))))$\\
        \textbf{Using closure to prove nonregular}: assume it is regular, use closure with known regular language, result is known nonregular language, so it must be nonregular, ex: $\#_a (w) = \#_b (w)$, intersect with $a^* b^*$ to get $a^n b^n$, intersection \& complement most useful\\
        \textbf{Octal divisible by 7}: 0, 7, 16, 25, \ldots, true only if sum of digits divisible by 7, so states in FSM are mod 7, is regular\\
        \textbf{Extras}:\\
        \boldmath$a^i b^j c^k; i, j, k >= 0$, if $i = 1$ then $j = k$\unboldmath: can change conditions to $i \neq 1 \vee j = k$, all strings with length $\geq 1$ is pumpable, so need to intersect with $a b^* c^*$, so $i = 1$ guaranteed and results in $a b^j c^k; j, k > 0 \wedge j = k$, then use pumping theorem\\
        \boldmath$a^i b^j, i \neq j$\unboldmath: must use $a^k b^{k+k!}$, if only use $a^k b^{k+1}$ can just pump 2 $a$ at a time to skip the equal part, $y = a^p$ for some nonzero $p$, pump in $\frac{k!}{p}$ times (must be integer because $p < k$), get $k + (\frac{k!}{p})p = k+k! $, alternatively prove that the complement is not regular using closure ($\neg L = a^n b^n \cup \{$out of order$\}$, intersect with $a^* b^*$ to get $a^n b^n$)\\
        \underline{\textbf{Context-Free Languages}}\\
        \textbf{Rewrite system}: list of rules \& algorithms for applying them, match LHS of some rule against some part of working string \& applies it, loop until told to stop, takes system \& initial string as input, if given grammar \& start symbol will generate language\\
        \textbf{CFG}: no restriction on RHS but LHS must still have 1 nonterminal\\
        \textbf{CFG Quadruple}: $(V, \Sigma, R, S)$, $R$ is finite subset of $(V - \Sigma) \times V^*$\\
        \textbf{CFL}: Language is context-free iff generated by some CFG\\
        \textbf{Recursive}: RHS can generate own LHS\\
        \textbf{Self-embedding}: recursive but also includes terminals on both sides afterwards, if not true then must be regular (not necessarily vice versa)\\
        \textbf{Concatenation}: use 2 nonterminals together in RHS, generate each separately\\
        \boldmath$a^n b^n$\unboldmath: $S \to aSb | \varepsilon$\\
        \textbf{Balanced parentheses}: $S \to \varepsilon | SS | (S)$\\
        \boldmath$w w^R$\unboldmath\textbf{(Even palindrome)}: $S \to aSa | bSb | \varepsilon$\\
        \boldmath$\#_a (w) = \#_b (w)$\unboldmath: $S \to aSb | bSa | SS | \varepsilon$\\
        \textbf{Arithmetic}: $S \to E + E | E * E | (E) | id$\\
        \boldmath${(a^n b^n)}^*$\textbf{(each region can have different $n$)}\unboldmath: $S \to MS | \varepsilon, M \to aMb | \varepsilon$\\
        \boldmath$a^n b^m: n \neq m$\unboldmath: $S \to A,B$ (more $a$ than $b$, more $b$ than $a$), $A \to a, aA, aAb$ (at least one extra generated), $B \to b, Bb, aBb$\\
        \boldmath$a^n b^m c^p d^q : m+n = p+q$\unboldmath: $S \to aSd | T | U, T \to aTc | V, U \to bUd | V, V \to bVc | \varepsilon$\\
        \textbf{Removing unproductive rules}: mark every nonterminal as productive \& every terminal as productive, if RHS all productive then mark LHS productive, loop until no changes, remove unproductive\\
        \textbf{Removing unreachable rules}: mark start as reachable \& all others as not, if LHS reachable mark all nonterminals in RHS as reachable, loop until no changes, remove unreachable\\
        \textbf{Parse trees}: leaf nodes are terminals except $\varepsilon$, root is start, all others are nonterminals\\
        \textbf{Branching factor}: longest RHS length\\
        \textbf{Generative capacity}: weak (set of strings), strong (set of parse trees)\\
        \textbf{Ambiguity}: multiple parse trees for 1 string, can come from different operator first in arithmetic orfrom rules $S \to SS | \varepsilon$, even if only 1 $S$ needed can generate multiple \& nullify\\
        \textbf{Reducing ambiguity}: remove null rules, recursive rules with symmetric RHS \& ambiguous optional postfix (ex: dangling else, can attach to inner or outer if)\\
        \textbf{Removing null rules}: mark all nonterminals in null rules as nullable, if any rule has nullable RHS then mark LHS as nullable, then for all modifiable rules (RHS has at least 1 nullable) add new rules containing variant without nullable, remove null rules\\
        \textbf{Removing symmetric rules}: force branching to 1 direction, replace $S \to SS$ with $S \to SS_1 | S_1$ for left branching, then make $S_1$ do what $S$ originally did\\
        \textbf{Operator order}: if parsed first (ex: in arithmetic, $E \to E + T$) then lowest priority\\
        \underline{\textbf{Chomsky Normal Form}}\\
        \textbf{Normal form}: for set C of data is a set of syntactically valid objects, for every element of C there is an equivalent element in F with respect to some set of tasks (possibly except finite exceptions), and at least some tasks are easier to perform on elements of F than C\\
        \textbf{CNF}: RHS has 1 terminal or 2 nonterminals, parsers can use binary trees, exact length of derivations known ($|w| - 1$ applications of nonterminal rules \& $|w|$ applications of terminal rules)\\
        \textbf{Conversion to CNF}: remove null rules, remove unit productions ($A \to B$), remove mixed rules (RHS length > 1 \& has terminal), remove long rules (length > 2)\\
        \textbf{Remove unit productions}: pick one $X \to Y$, remove, for every rule $Y \to \ldots$ make new rule $X \to \ldots$ unless it has already been removed once\\
        \textbf{Remove mixed rules}: make new nonterminal $T_a$ for each terminal $a$, for all qualifying rules substitute terminals for new nonterminals, then add rules $T_a \to a$ for all pairs\\
        \textbf{Remove long rules}: chain them, ex: $A \to BCDE$ becomes $A \to BX_1, X_1 \to CX_2, X_2 \to DE$\\
        \underline{\textbf{Pushdown automaton}}\\
        \textbf{Sixtuple}: $(K, \Sigma, \Gamma, \Delta, s, A)$, $\Gamma$ is stack alphabet, $\Delta$ is transition relation, finite subset of $(K \times (\Sigma \cup \{\varepsilon\}) \times \Gamma ^*) \times (K \times \Gamma ^*)$, state, input, pop, state, push\\
        \textbf{Configuration}: element of $K \times \Sigma^* \times \Gamma^*$, initial is $(s, w, \varepsilon)$\\
        \textbf{Top of stack}: leftmost character\\
        \textbf{Accepting}: in accepting state, stack \& input empty\\
        \textbf{Transitions}: input/pop/push\\
        \boldmath$a^n b^n$\unboldmath: push all $a$, when see $b$ start popping $a$ for every $b$, for $2n$ on $b$ push 2 $a$ for every $a$ in input\\
        \textbf{Balanced parentheses}: push opening, pop opening for every closing\\
        \boldmath$wc w^R$\unboldmath: push everything before $c$, after pop matching\\
        \textbf{Even palindrome}: like above, but nondeterministically decide where the middle is instead of using $c$ ($\varepsilon/\varepsilon/\varepsilon$)\\
        \boldmath$a^m b^n: m \neq n$\unboldmath: start with equal, if stack \& input empty reject, if leftover detected in either stack or input clear \& accept, still need to be sure of order\\
        \boldmath$\neg a^n b^n c^n$\unboldmath: out of order, $i \neq j, j \neq k$ (last two is just unequal $a, b, c$)\\
        \textbf{Deterministic}: iff $\Delta_M$ contains no pairs of transitions that complete with each other \& whenever $M$ is in accepting configuration never forced to choose between accept \& continue (via $\varepsilon$-transition with no popping)\\
        \textbf{Reduce nondeterminism}: use \# as bottom of stack marker \& \$ as end of string marker\\
        \textbf{CFG to PDA top down}: 2 states, start at p, q accepting, from p to q push start symbol, loop on q for each rule pop LHS and push RHS, for each terminal pop\\
        \textbf{CFG to PDA bottom up}: 2 states, start at p, q accepting, from p to q pop start symbol, loop on p for each rule pop reverse RHS and push LHS (reduce), for each terminal push (shift)\\
        \underline{\textbf{Pumping lemma for CF}}\\
        \textbf{Show CF}: create CFG, PDA, use closure\\
        \textbf{Definition}: $\exists k \geq 1 (\forall $strings $w \in L$, where $|w| \geq k (\exists u, v, x, y, z (w = uvxyz, vy \neq \varepsilon, |vxy| \leq k, \forall q \geq 0 (u v^q x y^q z$ is in L $))))$\\
        \textbf{Closures}: union (CFG with 2 possible paths), concatenation (above), Kleene star (let start repeat or $\varepsilon$), reverse (convert to CNF, flip nonterminal rules), interection with R, difference with R\\
        \textbf{Intersection}: try $a^n b^n c^m \cap a^m b^n c^n$, get $a^n b^n c^n$, 2 originals CF but result not, but intersection of CF \& R is CF (do it same way as intersection of R, works since only 1 stack needed)\\
        \textbf{Complement}: $a^n b^n c^n$\\
        \textbf{Difference}: $\neg L = \Sigma^* - L$, $\Sigma^*$ is CF, if closed under difference would be closed under complement, but just proved latter is false, but works for CF \& R ($L_1 - L_2 = L_1 \cap \neg L_2$)\\
        \boldmath$ww$\unboldmath: intersect with R, so $L' = L \cap a^* b^* a^* b^*$ to restrict form, then use pumping lemma on $L'$, if $L$ is CF then $L'$ must be CF but it is not\\
        \textbf{Deterministic CF Closures}: complement but not union \& intersection\\
        \textbf{Proof that nondeterministic CFL exists}: complement of $a^i b^j c^k: i \neq j \vee j \neq k$ is $a^n b^n c^n \cup$ out of order, but then intersecting with $a^* b^* c^*$ returns $a^n b^n c^n$, so original is CF but not DCF\\
        \underline{\textbf{Turing machine}}\\
        \textbf{Sixtuple}: $(K, \Sigma, \Gamma, \delta, s, H)$, $\Sigma$ does not contain $\square$, $\Gamma$ must contain $\square$ \& have $\Sigma$ as subset, $H \subseteq K$ is set of halting states, $\delta$ is transition function $(K - H) \times \Gamma$ to $K \times \Gamma \times \{ \to, \leftarrow \}$\\
        \textbf{Transition action}: go to next state, write, move head\\
        \boldmath\textbf{Convert $a^i b^j, 0 \leq j \leq i$ to $a^n b^n$}\unboldmath: mark $a$ with \$, find next $b$, if found mark with \# \& search for next pair, else write \# \& search for unmarked $a$, finally replace \$ to $a$ \& \# to $b$\\
        \textbf{Configuration}: $K \times ((\Gamma - \{\square\}) \Gamma^*) \times \Gamma \times (\Gamma^* (\Gamma - \{ \square \})) \cup \{\varepsilon\}$, state, active tape up to scanned square, scanned square, all active tape after scanned square\\
        \textbf{Start head position}: before input\\
        \textbf{Macros}: $M_x$ writes $x$, $L/R$ moves left/right once, $h$ halts, $y$ accepts, $n$ rejects, $x \leftarrow a,b$ label in transition means if read either then assign to variable $x$ \& continue, if part of $L/R$ then find first one \& store, $L/R_x$ find first $x$ to left/right of current\\
        \boldmath\textbf{Convert $w \in {\{1\}}^*$ to $w^3$}\unboldmath: mark each 1 with \#, go to end, put 2 \#, repeat until no 1, then convert all \# to 1\\
        \boldmath\textbf{Convert $u \square w$ to $uw$\unboldmath}: start from middle blank, for each character in 2nd half store in variable, write blank in original spot, write variable to the left\\
        \boldmath$a^n b^n c^n$\unboldmath: for each $a$ mark, then mark corresponding $b$ \& $c$, if no more $a$ verify no leftover $b$ or $c$, also verify order of characters\\
        \boldmath$wcw$\unboldmath: look at each character from 1st half \& compare to corresponding 2nd half using variables\\
        \boldmath$b^* a{(a \cup b)}^*$\unboldmath: at least 1 $a$, $b$ can come anywhere, so just scan for $a$\\
        \textbf{Computing}: TM computes a function iff for all $w$ in alphabet \& $w$ is an input on which $f$ is defined $M(w) = f(w)$, otherwise it does not halt, output of function is everything after head\\
        \textbf{Computable/Recursive}: iff TM exists that computes it and always halts\\
        \textbf{Copy machine}: for each character store in variable, write blank, $R_\square$ 2x (skip middle blank \& get to end of 2nd part), write variable, go back to start of 1st part \& write variable back\\
        \textbf{Binary +1}: go to end, flip all successive 1s at the end to 0s, then behind all the flipped bits write 1\\
        \boldmath$f(w) = ww$\unboldmath: copy, then shift 2nd part back\\
        \textbf{Busy beaver functions}: $S(n)$ is max steps executed by TM (with tape alphabet blank \& 1, halt on blank tape) with n non-halting states when started on blank tape before halting, $\Sigma(n)$ is max number of 1s left on tape by TM with n non-halting state before halt\\
        \textbf{Simulating multi-tape TM on single tape}: $O(n \times (|w| + n)) = O(n^2)$ assuming that $n \geq w$, $n$ is number of passes\\
        \textbf{Nondeterministic quintuple}: $(K, \Sigma, \Gamma, \Delta, s, H)$, $\Delta$ is subset of $((K-H)\times\Gamma) \times (K\times\Gamma\times \{ \to, \leftarrow \})$\\
        \textbf{2 of at least 1 letter}: nondeterministically select 1 letter, then see if there are two of them\\
        \underline{\textbf{Week 10}}:\\
        \textbf{Simulating real computer}:\\
        \textit{Tapes}: memory, program counter, address register, accumulator, current instruction opcode, input, output\\
        \textit{Formats}: memory is series of (address, value) pairs separated by delimiters, instruction is series of instruction number, 4-bit opcode \& address (no need to separate last 2)\\
        \textit{Algorithm}: Move input string to input file, initialise PC to 0, loop (scan memory for instruction matching PC, copy opcode to current instruction, copy address to address register, PC+1, scan memory for address in address register (operand), if load copy memory to accumulator, if add add operand to accumulator, if jump copy address register to PC, etc)\\
        \textit{Performance}: $O(n^3)$, $O(n^6)$ on single tape\\
        \textbf{Encoding TM}:\\
        \textit{States}: number them from 0 to $|K| - 1$ in binary, use smallest number of digits that fits, 0 for start, prefix with $y$ for accept, $n$ for reject, $h$ for halt, $q$ else\\
        \textit{Tape alphabet}: prefix with $a$, same numbering as states, 0 is blank\\
        \textit{Transitions}: (state, input, state, output, move), special case for start state that accepts: (q0)\\
        \textit{Final}: list of transitions\\
        \textbf{Universal TM}: on input $<M,w>$, $U$ must halt iff $M$ halts on $w$, if $M$ is deciding/semideciding then must also accept/reject accordingly, if $M$ computes function then $U(<M,w>)$ must equal $M(w)$\\
        \textbf{Running UTM}:\\
        \textit{Tapes}: $M$'s tape, $<M>$ (program to run), encoding of $M$'s state (program counter)\\
        \textit{Initialisation}: copy $<M>$ to program tape, write 1st state to state tape\\
        \textit{Simulation}: scan program for quintuple that matches current (state, input) pair, perform associated action, if no matching quintuple halt, finally report result\\
        \textbf{Church-Turing Thesis}: all formalisms powerful enough to describe everything we think of as computational algorithm are equivalent\\
        \underline{\textbf{Week 11}}\\
        \textbf{TM Universe Definition}: only strings which meet syntactic requirements of the language definition, set of syntactically valid strings is in D\\
        \textbf{Complement}: with respect to universe of the same syntactic form, ex: complement of $<M,w>$: TM $M$ halts on input $w$ is still in the form $<M,w>$, but simply does not halt\\
        \textbf{Halting problem is SD}: run the TM, then accept, if no halt will never accept\\
        \textbf{Halting problem is not D}: contradiction, if decidable there will be TM to do so (if halt accept, else reject), but if $M_H(<Trouble, Trouble>)$ is run ($Trouble(x)$: if $M_H$ accepts $<x,x>$ then loop forever, else halt) then if $M_H$ returns true $Trouble$ loops, but if return false then $Trouble$ does halt\\
        \textbf{If H is in D all SD would be in D}: there will be semideciding TM, if H in D there will be oracle $O$ to decide it, then we can run $O$ on $<M_L,w>$, if $O$ accepts (will halt), then run, if it accepts then accept, else reject, if will not halt then reject\\
        \textbf{Set of CFL is proper subset of D}: CFLs decidable, so subset, \& at least 1 language is decidable but not CF ($a^n b^n c^n$)\\
        \textbf{There are languages not in SD}: countably infinite number of SD (enumerate TM) over nonempty alphabet, but uncountably infinite languages (power set)\\
        \textbf{SD extra example}: email of someone who will respond to message in group\\
        \textbf{SD not closed under complement}: if can semidecide both it and complement then can decide it, would be in D, contradiction\\
        \boldmath\textbf{$\neg H$ is not in SD}\unboldmath: $H$ is in SD but not D, if $\neg H$ in SD then $H$ would be in D, but it's not, so can't be in SD\\
        \textbf{Languages not in D}: $H$, $\neg H$, does $M$ halt on empty tape, is there any string on which $M$ halts, does $M$ accept all strings, do 2 TMs accept same language, is language accepted by TM regular\\
        \textbf{Rice's theorem}: no nontrivial property of SD languages is decidable, nontrivial is one that is not simply true for all languages or false for all\\
        \textbf{Using Rice's theorem}: specify property P, show that domain is SD, show it is nontrivial\\
        \textbf{Is language regular}: is $\{<M>: L(M)$ is regular$\}$ in D, P is true if $L$ is regular, false otherwise, domain is SD since it is languages accepted by TM, nontrivial because $a^*$ is regular but $a^n b^n$ is not\\
    \end{multicols}
    \end{document}
