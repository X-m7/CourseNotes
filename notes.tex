\documentclass[a4paper]{article}
    \usepackage[margin=0in]{geometry}
    \usepackage{multicol}

    \begin{document}
    \begin{multicols}{3}
        \tiny
        \noindent \textbf{Introduction} \underline{What is HCI} - interaction btw ppl + comp-based sys, concerned w phys, psyc + theoretical aspects of this process, designing comp sys that support ppl so
        they can carry out act productively + safely;
        \underline{usr} - indiv usr, group of usrs working tgr or seq of usrs in org dealing w part of process/task;
        \underline(computer) - tech ranging from desktop to large scale sys or control/embedded sys;
        \underline{interaction} - commun. btw usr + comp in direct/indirect manner;
        \underline{what is involved} - study of hmns using interf, dev of new apps/sys to support usr's act, dev of new devices + tools for usrs; 
        \underline{interdisciplinary subj} - psyc, ergonomics, sociology involves, comp science + sys design are central concerns, not possible to design effec interactive sys from 1 discipline in isolation;
        \underline{importance of HCI} - need to know how to make sys usable, need to be able to eval usability of custom made + COTS sys, good deisgn involves understing how
        usrs interact w comp + enabling usrs to do so effectively; is inc matter of law, national health + safety stanfarfs constrain employers to provide workforce w usable comp sys;
        designers + employers cannot afford to ignore usr;
        \underline{Dev process} - design approaches, implem techniques \& tools, eval techniques, example sys + case studies;
        \underline{Prob w SW} - SW blamed for excessive + unwanted share dealing on world's stock market, errors in dosages given to patents receiving radiation therapy, erratic behaviour of military + civil aircraft;
        \underline{Avoiding problematic design} - take into acct: who usrs are, what act are carried out, where interac taking place; optimise interac usrs have w sys such that they match usrs' act + needs;
        \underline{Use Words} - useful (accomplish what's req), usable (do it easily + naturally, w/o danger or errors), used (make ppl want to use it, be attractive, engaging, fun etc.);
        \underline{Principles for supporting HCI} - taking into acct what ppl are good + bad at, considering what might help ppl w way they currently do things,
        listening to what ppl want + getting them involved in design processes; \newline \textbf{Interaction Design} \newline
        \underline{Good design} - based on how everyday obj behave, easy + intuitive + pleasure to use, only req 1-step actions to perform core tasks;
        \underline{What To Design} - take into acct: who usrs are, what act being carried out, where interac taking place, need to optimise interactions usrs have w product so they match usrs' act + needs;
        \underline{understing usrs' Needs} - need to take into acct what ppl good + bad at, consider what might help ppl in way they curr do things, think through what
        might provide quality usr exp, listen to what ppl want + get them involved, use tried + tested usr-centered methods;
        \underline{Interaction design (ID)} - "designing interac prod to support way ppl commun + interact in their everyday + working lives", "design of spaces for hmn commun + interac";
        \underline{Goals of Interac Design} - dev usable prod; usability means easy to learn, effective to use + provide enjoyable exp; involve usrs in design process;
        \underline{Which kind of design} - no. of other terms used emphsising what's being designed (eg. UI design, SW design, UX design etc.), interac design is umbrella term covering all aspects;
        fundamental to all disciplines, fields + approaches concerned w researching + designing comp-based sys for ppl;
        \underline{Relation btw ID, HCI + other fields} - acad desciplines contrib to Interac Design: psy, social sciences, computing, eng, ergonomics, informatics;
        design prac contrib to ID: graphic design, prod design, artist-design, indus design, film induc; interdisciplinary fields 'do' interac design: HCI, ubiquitous comp, hmn factors, cogn eng,
        cogn ergonomics, comp supported co-operative work, info sys;
        \underline{Working in Multidisc teams} - many ppl from diff bckgrounds involved, diff perpec + ways of seeing + talking about things; benefits (more ideas + design gen); disadv (hard to commun
        + progradd forward designs being created);
        \underline{What do prod do in ID business} - interac designers: ppl involved in design of interac aspects of pro, usability eng: ppl who focus on eval prod
        using usability meth + principles, web designer: ppl whoo develop + create visual design of webs such as layouts, info architects: ppl who come up w ideas of how to plan +
        struc interac prod, UX designers - ppl who do all prev stated things but who may also carry out diels studies to inform design of prod;
        \underline{usr Exp} - how prod behaves + is used by ppl in real world, way ppl feel about it + their pleasure + satisfac when using it, looking at it, holding it + opening/closing it;
        cannot design UX, only design for UX;
        \underline{Process of ID} - est req, dev alt, prototyping, eval;
        \underline{Char of ID} - usrs should be involved through dev of project, specific usability + UX goals need to be iden., clearly doc + agreed at beginning of project;
        iteration needed through core act
        \underline{Why ID} - help designers: underst how to design interac prod tha fit w what ppl want, need + may desire; appreciate that one size doesn't fit all, id
        any incorrect assump they may have about part usr groups; be aware of both ppls sensitivities + capabilities;
        \underline{Accessibility} - degree to which prod usable + accessible by as many ppl as possible, focus on disability;
        \underline{Usability goals} - effective to use, effic to use, safe to use, have good utility, easy to learn, easy to remember how to use;
        \underline{UX Goals} - desirable aspects: satisfying, helpful, fun, enjoyable, motivating, provocative, challenging etc., undesirable aspects: boring unpleasant, annoying, cutesy etc.
        \underline{Usability + UX Goals} - selecting terms to convey person's feels, emotions etc., can help designers underst multifaceted nature of UX;
        \underline{Design Principles} - gen abstrac for thinking about diff aspects of design, dos + don'ts of interac designs, what to provide + what not to prov at
        interf; derived from mix of theory-based knowledge, exp + common-sense;
        \underline{Feedback} - sending info back to usr about what has been done; includes sound, highlighting, animation + combo of these (eg when screen
        button clicked on prov sound/highlight feedback);
        \underline{Constraints} - restr possible actions that can be perf, helps prev usr from selec incor options, physical obj can be designed to constrain things (eg. only 1 way to insert key into lock)
        \underline{Logical + Ambiguous design} - L: prov direct adj mapping btw icon + connector, consistency: desugn interf to have similar oper + use similar elem for similar
        task, main benefit consistent interf easier to learn + use; when consistency breaks down: what happens if theres more than 1 command starting w same latter $\to$ have to find
        other initials + combo of keys thereby breaking consistency rule, inc learning burden on usr making them more prone to errors; internal + external consistency: internal
        cons. - designing oper to behve same within app (hard to achieve w complex interf); external consis refers to designing oper, interf etc., to be same across apps +
        devices (very rarely the case based on diff designer's preference);
        \underline{Affordances} - refers to attr of obj that allows ppl to know how to uae it (eg mouse button invites pushing, door handle affords pulling), since has been much
        popularised in ID to discuss how to design interf obj (eg scrollbars to afford moving up + down, icons to afford clicking on); Affordances + ID: interf virtual + don't have
        affordances like physical obj; interf better conceptualised as 'perceived affordances' (learned conventions of arbitrary mappings btw action + effect at interf, some mappings
        better than others); \underline{Key points} - ID concerned w designing iterac prod to support way ppl communicate + interac in everyday lives; concerned w how to create
        quality usr exp, req taking into acct no. of interdep factors (contect of use, type of act, cultural diff, usr groups), multidisciplinary \newline \textbf{understing + Conceptualising Interac}
        \underline{assumption} - taking something for granted when it needs firther investig., realistic or wishlist?; \underline{Claim} - stating something to be true when it's still open to q;
        \underline{Framework for Analysing problem space} - are there probs w existing prod/usr exp? If so what are they? How will your proposed design will overcome these? If you're designing for a
        new usr exp how do you think your proposed design ideas support, change or extend curr ways of doing things? \underline{Benefits of Concep} - orientation (enables design teams to ask specific
        q about how concep model will be understood), open-minded (prevents design teams from becoming narrowly focused early on), common ground (allows design teams to est set of commonly agreed terms);
        \underline{From problem space to design space} - having good understing of problem space can help inform design space, before deciding upon these it's important to dev concep model; \underline{Conceptual Model} -
        "a high-level desc of how sys org + operates"; enables designers to straighten out thinking before starting to lay out widgets; \underline{Components} - metaphors + analogies (underst what prod is for
        + how to use it for activity), concepts ppl exposed to through prod (task-domain, their attr, oper (eg saving, revisiting, organising), relo + mappings btw these concepts);
        \underline{First steps in formulating conceptual model} - what will usrs be doing when carrying out tasks?, how will sys support these tasks?, what kind of interf metaphor (if any) will be appropriate?,
        what kinds of interac modes + styles to use?; \underline{interf metaphors} - concep what to do, a concep
        model instantiated at interf, visualising oper; interf designed to be similar to physical entity but also has prop, can be based on act/obj/combo of both, exploit usr's familiar
        knowledge to help them underst the unfamiliar; \underline{Benefits of interfact metaphors} - makes learning new
        sys easier, helps usrs underst underlying concep model, can be innovative + enable realm of comp + their apps to be made more accessible to greater diversity of usrs; \underline{Problems w interf Metaphors} -
        break concentional + cultural rules, can constrain designers in way they conceptualise problem space, conflict w design principles, forces usrs to only underst sys in terms of metaphore, designers can
        inadvertently use bad existing designs + transfer bad parts over, limit designers' imagination in coming up w new concep models; \underline{Interac Types} - \underline{instructing} (issue commands + select options, where usrs instruc sys
        + tell it what to do, very common (eg word processors, HDRs, vending machines, main benefit is it supports quick + effic interac)), \underline{conversing}
        (interacting w obj in virtual/physical space by manipulating them, underlying model of having convers w another hmn, range from simple voice rec menu driven sys to more complex 'natural lang' dialogs,
        eg advice-giving sys, timetables, search engines, help sys, pros: allow usrs (esp novices + technophobes) to interact w sys in way that's familiar, misund arise when sys doesn't know to parse what usr says);
        \underline{manipulating} (interac w obj in virtual/phys space, (eg dragging, selecting, opening, closing + zooming actions on virtual obj), exploit usr knowl of how they move + manipulate phys world, can involve phys controllers or
        air gestures (eg. wii, kinect), tagged phys obj (eg balls), \underline{direct manip} (continuous rep of obj + actions of interest, phys actions + button pressing instead of issing commands w complex syntax,
        rapid reversible actions w immediate feedback on obj of interest), adv of DM (novices can learn basic func quickly, exp usrs can work extremenly rapidly to carry out wise range of
        tasks, even defining new func, intermittent usrs can retain oper concepts over time, error msges rarely needed, usrs can immed see if their actions are furthering their goals, usrs exp less anxiety, usrs gain conf
        + mastery + feel in control), disadv of DM (some ppl take metaphore of DM too literally, not all tasks can be desc by obj + not all actions be done directly, some tasks are better achieved through
        delegating (eg spell checking), can become screen space 'gobblers', moving mouse around screen can be slower than pressing func keys to do same actions)); \underline{exploring} (moving through virtual
        enciron/phys space, involves usrs moving through virtual/phys environ, phys environ w embedded sensor tech); \underline{Which concep model best?} - DM good for 'doing' types of tasks (eg designing, drawing, flying, driving,
        sizing windows), issuing instruc good for repetitive tasks (eg spell-checking, file manag), having convers good for children, comp-phobic, disabled usrs, spec apps (eg phone services), hybrid concep model odten employed
        where diff ways of carrying out same actions supported at interf but can take longer to learn; \underline{Interaction + interf} - interac type: what usr is doing when interac w sys (eg instruc,
        talking, browsing), interf type: kind of interf used to support mode (eg speech, menu-based, gesture, query, data-entry, web); Which interac type to use? need to determine req + usr needs, take budgest + other
        constraints into acct, also will depend on suitability of tech for act being upported; \underline{Paradigm} - insp for concep model, general approach adopted by commun for carrying out research (shared assump,
        concepts, values + practices, (eg desktop, ubiquitous comp, pervasive comp, wearable comp, tangible bits, augmented reality, attentive envrion, ambient comp, in wild)); \underline{Visions} - driving force that R\&D,
        provide concrete scen of how society can use next gen of imagined tech, raise many qd concerning privacy + trust; \underline{Theory} - explanation of phenomenon (eg info processing that explains how mind, or some aspect
        of it, assumed to work), can help id factors (eg cognitive, social, affective, relevant to design + eval of interac prod); \underline{Models} - simplif. of HCI phenomenon (intended to make it easier for designers to
        predic + eval alt designs, abstrac from theory coming from contrib discipline (eg psychology, eg keystroke model); \underline{Framework} - set of interrelated concepts and/or spec qs for 'what to look for', many in
        interac design (eg norman's concep models, benford's trajectory), provide advice on how to design (eg steps, qs, concepts, challenges, principles, tactics + dimensions); \newline \textbf{Cognitive Aspects} \newline
        \underline{Why do we need to underst usrs?} - interac w techn is cognitive proc involved + cognitive limitations of usrs, provides knowledge about what usrs can + cannot be expected to do, ids + explains nature + causes of
        probs usrs enounter, supply theories + modelling tools + guidance + methods that can lead to better interac prod; \underline{Cogn Proc} - 1. \underline{attention} (selecting things to concen on at point in time from mass of stimuli
        around us, allows us to focus on info that's relevant, involves audio and/or visual senses, focussed + divided attention enabkes us to be selec in terms of mass completing stimuli but limits our ability to keep
        track of all events, info at interf sgoyld be struc to capture usrs' attention (eg use perceptual boundaries (windows), colour, reverse videp, sound, flashing lights), \underline{Multitasking + attention} -
        multitaskers easily distracted + find it hard to filter irrel info, \underline{Design implecations for attention} - make info salient when it needs attending to, use techniques that make things stand out like colour,
        ordering, spacing, underlinking, sequencing + animation, avoid cluttering interf w too much info, eg search engines + form fill-ins that have simple + clean interf easier to use); 2. \underline{percep} (how info
        aquired from world + transformed into exp, obvious implic is to design rep that are readily perceivable (eg text should be legible, icons should be easy to distinguish + read), design implications (icons should
        enable usrs to readily disting their meaning, bordering + spacing effec visual ways of grouping info, sounds should be audible + distinguishable, speech output should enable usrs to distingish between set of spoken wordsm
        text should be legible + distinguishable from background, tactile feedback should allow usrs to recog + distinguish diff meanings)); 3. \underline{memory} (involves first encoding + retrieving knowledge, we don't
        remember everything - involves filtering + processing what is attended to, context important in affecting our memory (ie where, when), we remember less about obj we habe photographed than when we observe them w
        naked eye), \underline{Processing in mem} - encoding is first stage of mem (determines which info attended to in encuron + how it's interpreted, the more atten paid to something $\to$ the more its processed
        in terms of thinking bout it + comparing it w other knowledge $\to$ the more likely it is to be remembered, context is important: context affects extent to which info can be subsequently retrieved, sometimes it
        can be hard for ppl to recall info that was encoded in a diff context, recog vs recall: command-based interf req usrs to recall names, GUIs provide MP3 players visually-based options
        that usrs need only browse through until they recog one, web browsers etc provide lists of cisited URLs sone titles etc, that support recog mem, 7+/-2: ppl's immediate mem capacity very limited (useful finding
        for ID), small no. of things not relevant when ppl scan lists of bullets + tabs + menu items for one they want, they don't have to recall them from mem; \underline{digital context manag}: growing prob for usrs bc of vast
        no. of docs, images, music files, videp clips, emails, attachments, bookmarks etc, where + how to save them all then remembering what were called + where to find them again, naming most common means of encoding them,
        but can be hard to remember, mem invoves 2 proc (recall-directed + recog-based scanning), file manag sys should be designed to optimize both kinds of mem proc (eg search box + history list), help usrs encode files
        in richer ways (provide them w ways of saving files using colour, flagging, image, flex text, time stamping etc); digital forgetting: when want to forget something that's online; memory aids:
        SenseCam (microsoft research labs), wearable devide that intermittently takes photos w/o any usr intervention while worn, digital images taken are stored + revisited using special software, has been found to improve
        ppl's mem suffering from Alzheimers; design implic: don't overload usrs' mem w complic proced for carring out tsks, design interf that promo recog than recall, provide usrs w various ways of encoding
        info to help then remember (eg categories, colour, flagging, time stamping)); 4. \underline{learning} (how to learn to use comp-based app, use comp-pased app or youtube vid to underst given topic,
        ppl find it hard to learn by following instruc in manual (prefer to learn by doing); cognitive prosthetic devices: rely more on internet + smartphones to look things up, expecting to have internet
        access reduces need + extent to which we remember; design implic: design interf that constrain + guide learners, dynamically link concepts + rep can facil learning of complex material); 5.
        \underline{reading, speaking + listening} (ease w which ppl can read, listen or speak differs; many prefer listening to reading, reading can be quicker than speaking or listening, lisening req less cognitive
        effort than reading or speaking, dyslexics have hardies understing + recog written words; apps: speech-recog sys allow usrs to interact w them by asking qs (eg googl voice, siri), speech-output sys
        use artificially gen speech (eg written-text-to-speech sys for blind, natural lang sys enable usrs to type in qs (eg ask search engine); design implic: speech-based menus + instruc should be short, accentuate
        intonation of artificially gen speech voices (harder to underst than hmn voices), provide opport for making text large on screen)); \underline{Problem-solving, phanning, resource + decision-making}
        (all involves reflec cogn (eg thinking about wha to do, what options are \& consequences), often involves conscious proc + discussion w others (or oneself), + use of artefacts (eg maps, books, pen +
        paper); may involve working through diff scenarios + deciding which is best option; design implic: provide more info/func for usrs who wish to underst more about how to carry out act more effect.,
        use simple comput. aids to support rapid decision-making + planning for usrs on the move; dilemma: \underline{app mentality} dev in psyche of younger gen making it worse for them to make own decisions bc they're
        becoming more risk averse, replying on multitude of app means that they are inc more anxious about making decisions by themselves; \underline{mental model}: usrs develop underst of of sys through learning
        bout + using it, knowledge sometimes desc as mental model (how to use sys (what to do next), what to do w unfamiliar/unexpec situs (how sys works), ppl make interences using mental
        models of how to carry tasks), desc as internal construc of some aspect of external world enabling predic to be made, involves unconscious + conscious proc (images + analogies activate),
        deep vs shallow model (eg how to drive car + how it works), example of bad mental model: setting thermostat to be at its highest (instead of desired temp) to warm up house as quickly as possible);
        \underline{Gulf of Execution + Eval} - 'gulfs' explicate gaps that exist btw usr + interf, \underline{gulf of execution} (distance from usr to phys sys), \underline{gulf of eval} (distance from phys sys to usr),
        bridging gulfs can reduce cognitive effort req to perform tasks; info proc: concep hmn perf in metaphorical terms of info proc stages; \underline{Model hmn Proc}: models info proc of usr interac w comp,
        predicts which cognitive proc are involved when usr iteracts w comp, enables calcs to be made of how long usr will take to carry out task; \underline{Limitations}: based on modellling mental act that
        happen exclusively inside head, doesn't adequately acct for how ppl interact w comp + other devices in real world; \underline{Distributed cong} - concerned w nature of cognitive phenomena across
        indiv, artefacts + internal + external rep; desc these in terms of propagation across rep state; info transformed through diff media (comp, displays, paper, heads), diff btw this + info proc: info
        proc is talking about 1 mind, distrib is when working with other ppl; \underline{What's involves w DC} - dist prob-solving that takes place, role of verbal + non-verbal behaviour, various coordinating
        mech that are used (eg rules, proced), commun that takes place as collaborative act progresses, how knowledge is shared + accessedl \underline{External Cognition} - concerned w explaining
        how we interac w external rep (eg maps, notes, diagrams); what are cognitive benefits + what proc involved; \underline{Externalizing to reduce mem load} - diaries, reainders, calendars, notes,
        shopping lists, to-do lists (written to remind us of what to do); post-its, piles, marked emails (where places indicates priority of what to do); external rep - remind us that we need to do
        something (eg to buy something for morther's day), remind us of what to do (eg buy a card), remind us when to do something (eg send a card by a certain date); \underline{Computational offloading}
        - when tool used in conjuction w external rep to carry out comp (eg pen + paper); \underline{Annotation + cognitive tracing} - annotation involves modifying existing rep through making marks (eg crossing
        off, ticking, underlining); cogn tracing involves externally manip items into diff orders/struc (eg playing Scrabbles, playing cards); \underline{Design imp} - provide external rep at interf that red mem load
        + facil computational offloading (eg info visuali. designed to allow ppl to make sense + rapid decisions about masses of data); \newline \textbf{Design, prototyping and construction} -
        prototype: prototype is small-scale in other design fields (miniature car, miniature building/town), Interac design prototypes (series of screensketches, storyboard, slideshow, videp); \underline{Why prototype} -
        eval + feedback are central interac design; stakeholders can see, hold, interac w prototype more easily than soc/drawing; team mem can commun effectively; can test out ideas for yourself; encourages
        reflec: important aspect of design; prototypes answer qs + support designers in choosing btw alt; \underline{Dimensions of prototyping} - appearance (size, colour), data (data size, data type), functionality (sys func, usr func),
        interac (input behaviour, output behaviour), spatial struc (arrangement of interf or info elem); \underline{Manifestation dimensions of Prototyping} - material (phys media, eg paper, wood), resolution (accur
        of perf, eg feedback time), scope (lvl of contextuali. eg website colour scheme); \underline{What to prototype} - technical issues, work floc + task design, screen layouts + info
        display, difficult + controversial + critical areas; \underline{Low-fidelity prototyping} - uses medium that's unlike final medium (eg paper, cardboard); quick, cheap, easily changed (eg sketches of screens, task seq,
        storyboards etc.); \underline{Storyboards} - often used w scenarios, series of sketches; \underline{"Wizard-of-oz" Prototyping} - usr thinks they're interac w comp but dev resp to output rather than sys,
        done early to underst usr expec; \underline{High-fidelity Prototyping} - uses material you'd expect to be in final prod, prototype looks more like final sys than low-fidelity version, high-fidelity proto can
        be dev by interg existing HW + SW components; danger that usrs think they have complete sys; \underline{Compromises in Prototype} - two types of compromise (horizontal - provide wide range of func but
        w little detail, vertical - provide lot of detail for only few func); compromises in protot can't be ignored $\to$ prod needs eng; \underline{Low vs High fid} - adv LF: lower dev cost, eval mult design cons;
        disadv LF: limited error checking, facilitator-driven, navig + flow limtations; adv HF: complete func, fully interac, usr-driven; disadv HF: more resource-intensive, time consuming, not effec for req gathering;
        \underline{Concep design} - transform usr req into concep model, consider alt: prototyping helps; \underline{Eval metaphors} - how much struc does it prov?, how much is relev to prob?, is it easy to rep?,
        will audience underst it?, how extensible is it?; \underline{Condifering Interac + interf types} - which interac type? (how usr invokes actions; instruc, conversing, manip or exploring), do diff interac
        types provide inside? (shareable, tangible, augmented reality etc.); \underline{Expanding initial concep model} - what func will prod perform (what will prod do + what will hmn do); how are func related?
        (seq or parallel?; categorisations eg all actions related to privacy on smartphone), what info is needed? (what data req to perform task; how this data will be transformed by sys);
        \underline{Concrete design} - usr char + context: accessibility, cross-cultural design; cultural site guidelines: successful prod "are... bundles of social sol, inventors succeed in
        part culture bc they underst values institu. arragements + economic notions of that culture"; \underline{Using Scenarios} - express proposed/imagined situs, used throughout design in var ways (as
        basis for overall design, scripts for usr eval of prototypes, concrete example of tasks, as means of co-oper across pro boundaries); \underline{Explore usr's exp} - use personas, card-based
        prototypes/stickies to model usr exp; visual rep called (design map; customer/usr journey map, exp map), 2 common rep: wheel + timeline; \underline{Construc: phys comp} - build + code prototypes using
        electronics; designed for use by wide range of ppl; \underline{Construc: SDKs} - SW dev kits (SDKs); \newline \textbf{Affective Aspects + internat. usr interf} \underline{Emotions + usr exp} - HCI
        trad. been about designing efficient + effective sys; now more about how to design interac sys that make ppl rep in certain ways; \underline{Emotional Interac} - translate emos into diff aspects of usr
        exp.; why ppl become emo attaches to certain prod (eg virtual pets, how to change hmn behavior through use of emotive feedback); \underline{Claims from model} - emo state chanes how we think (angry $\to$ more likely
        to be less tolerant, happy $\to$ more likely to overlook minor probs + be more creative); \underline{Expressive interf} - provide reassuring feedback that can be both informative + fun (can also be intrusive
        causing ppl to get annoyed + angry)l colour, icons,  graphical elem + animations used to make 'look + feel' of interf appealing(conveys emo state); in turn this can affect usability of interf (ppl
        prep to put up w certain aspects of interf (eg slow download rate) if end result is appealing + aesthetic); \underline{Friendly interf} - miscrosoft pioneered friendly interfs for technophobes ('At home w
        Bob' SW), makes usrs feel more at easse + comfortable; \underline{Frustrating interf} - many causes (when app doesn't work properly/crashes, when sys doesn't do what usr wants it to do, when
        usr's exp aren't met, when sys doesn't provide suffic info to enable usr to know what to do, when error msges pop up that are vague/obtuse/condemning, when appearance of interf is
        garish/noisy/gimmicky/patronizing, sys req usrs to carry out too many steps to perf task only to discover mistake was made earlier + they start all over again); \underline{Gimmicks} -
        amusing to designer but not usr (eg clicking on link to site only to discover that it's still 'under construc'); \underline{Error Msges} - Shneiderman's guidelines for error msges (avoid
        using terms like FATAL, INVALID, BAD or audio warnings; avoid UPPERCASE + long code no.; msges should be precise rather than vague; provide context-sensitive help); \underline{Detecting Emos
        + Emo Technology} - sensing tech used to measure facial exp, gestures, body movement; aim to predict usr's emos + aspects if their behavior (eg what is someone most
        likely to buy online when feeling sad, bored or happy); \underline{Facial Coding} - measures usr's emos as they interac w comp or tablet; analyses imges captured by webcam of their face; uses this to
        gauge how engages usr is when looking at movies, honline shopping sites + ads; 6 core exp: sadness, happiness, disgust, fear, surprise + anger; \underline{How to use emotional data} - site can
        adapt its ad, movie storyline or content to match usr's enotional state; \underline{Detection} - beginning to be used more to infer or predict someone's behaviour; eg determining person's suitability
        for job or how they'll vote at an election; \underline{Persuasive technologies + behavioural change} - interac comp sys deliver. designed to change ppl's attitudes + behaviours; techniques to change what
        ppl thing (pop-up ads, tidy street project (enc ppl to reduc elec consump), warning msges, remainders, prompts, personalised msges, recom., Amazon 1-click, nintendo's pocket pikachu); ref to as nudging; \underline{Tracking devices} - mobile
        apps designed to help ppl monitor + change their behaviour (eg fitness, sleeping, weight); can compare w online leader boards + charts, to show how they've done in relation to their peers + friends; apps
        that encourage refl that in turn inc well-being + happiness; \underline{Anthromorphism} - attr hmn-like qualities to inanimate obj (eg cars, comp); well know phenomenon in advertising; much
        exploited in HCI (make usr exp more enjoyable, more motivating, make ppl feel at ease, reduce anxiety); evidence to support: Reeves + Naas found comp that flatter + praise usrs in educ SW prog had
        +ve impact on them; criticisms: deceptive, make ppl feel ancious, inferior, stupid; many prefer more impersonal "incorrect, try again."; \underline{Virtual Char} - appearing on our screens (sales agents,
        char in videpgames, learning companions, wizards, pets, newsreaders); provides persona that's welcoming, has personal. + makes usr feel involved w them; Disadv: can lead ppl into false sense of
        belief enticing them to confide personal secrets w chatterbots; may not be trustworthy; \underline{Believable Virtual Agent} - believeability refers to expect to which usrs come to believe agent's
        intentions + personality; \underline{Issues} - factors that effect indiv's response to sys (age. gender, race, sexuality, class, religion + political persuation); exported SW freq req
        modific. (to suit local customs, laws, conventions); blend technical + social facilities (eg groupware) $\to$ inc complexity of design issuesl dev of mult interf costlyl important
        to make generic + easily modif. interf as possible; \underline{3 Levels of interf Specialisation} - globalisation (applying allegedly cultureless standard; translation, gov regul,
        seamless intergration, business prac, cultural elem, brand manag); internat.isation (designing base struc w intent of later customis.); localization (dev specific interf to meet
        part market); \underline{Effective Design} - ED involves recog cultural elm in given app; cultural diversity makes it even more unrealistic for designers to rely on intuition or
        personal exp for interf design; adap of shared interf req id of usr factors including (obj factor: gender, age, ethnic background, mother-tongue; subj factors: which can't be directly
        measured/id'd cognitive style); \underline{Approaches} - adop of usr-centred dev approach (usrs + dev work tgr on id of factors which effect usability + which spec perf); effec
        use of iterative + parallel prototyping (to facil usr partip + max effectiveness of interf eval process while minimi time req for dev); Interfration of Taguchi techniques
        (privide rigor necessary for id of optimum interf - minimising var as main means of improving quality); systematic + logical interg of techniques (so method can be
        applied by commercial interf designers); \underline{Cultural Factors} - cultures could be reviewed through 4 dimensions (power distance: degree of dep btw book + subordinate;
        collectivism-individualism: integration into cohesive groups vs being expected to look after him/herself; femininity-masculinity: extent to which gender roles are distinct or overlap;
        uncertainty avoidance: extent to which members feel threatened by uncertain unknown situs); \underline{Cultural Meta Models} - cultural models help us underst how + where
        culture comes to influence our lives in profound way; model tells us  culture has multiple layers + what we can normally observe is only most external layer that counts for 10\%
        of its total influence; \underline{Iceburg Model} - 10\% of cultural char of target aud "easily visible" to observer; surface: visible, obvious rules such as no., currency,
        time + date formats; unspoken rules: obscured, need context of stiu to underst rules; unconscious rules: rules out of conscious awareness + difficult to study;
        \underline{Char sets + collating seq} - CSeq: define value + pos of each char w raspect to other char; CSets: alphabets; \underline{Format Conventions} - data formats, currency formats,
        time formats; \underline{Layout conventions}; \underline{Icons, symbols, colours; screen text} - 'x' or tick for checking boxes, red cross to id first aid; \underline{Menu accelerator}
        - shortcuts (eg cmd+C to copy bc copy starts w 'c'); \underline{Design Considerations} - id lang + country of usr $\to$ addr gov reg $\to$ support char sets $\to$ internat.
        database design $\to$ Code FORMs so data isn't corrupted $\to$ dispkay info in culturally correct ways $\to$ use approp currency $\to$ design graphical imges w care $\to$ provide alt
        to natural lang proc + audio/videp elem; \underline{Single App Localisation} - handle mult lang (ie in same app); universal text econding using Unicode; consistent but localised UI;
        \underline{Things to Avoid} - don't: hard code text in src code; refer to cultural specific standards; use slang, jargon, humour, sarcasm, colloquilism, metaphor (hard to translate, req
        understing of originating nulture); from plurals by adding "(s)" to indicate either singular or plural form (use both forms if necessary); \underline{usr Customisation} - krrp
         sentences as short + simple as possible; allow usrs to select date + time format; allow usrs to select calendar format; allow usrs to select paper size; allow usrs to select numeric +
         monetary formats; \newline \textbf{Usability Testing} UT - involves recording perf of typical usrs doing typical tasks; controlled seetings; usrs are observed + timed; data
         recorded on videp + key presses logged; data used to calculate perf times +, to id + explain errors; usr satif eval using ques + interviews; field obsers may be
         used to provide contextual understing; \underline{Experiments + Usability Testing} - exp test hypothesis to discover new knowledge by invest relo btw 2/more var; usability testing is
         applied experim; developers check that sys is usable by intended usr popul for their taks; UT: improve prod, few partic, results inform design, usually not completely replicable, cond
         controlled as much as possible, proced planned; Exp: discover knowledge, many partic., results validated statistically, must be replicable, strongly controlled cond, exper design; UT: goals + qs
         focus on how well usrs perf tasks w prod; compar of prod/prototypes common; focus on time to complete task + no. + type of errors; \underline{Testing Cond} - usability lab or other controlled space; emphasises (selecting rep usrs, dev rep tasks);
         5-10 usrs typically selected; tasks usually around 30 mins; test cond same for every particip; informed consent form explains proced + deals w ethical issues; \underline{Types of data} - time to
         complete task; time to compplete task after spec time away from prod; no. + type of errors per task; no. of errors per unit of time; no. of times online help + manuals accessed; no. of usrs
         seccessfully fin task; UT: some say testing should cont til no new insights gained; \underline{Problems + Actions} - Prob detected: accessing web hard; lack of affordance + feedback; getting lost;
        knowing where to tap; actions by evaluators: reported to dev; made avail to public; accessibility for all usrs important; \underline{Experimental designs} - diff partic (single group of partic allocated
        randomly to exp cond; adv: no order effects; disadva: many subj + indiv diff prob); same partic (all appear in both cond; adv: few indiv, no indiv diff; disadv: counter-balancing needed bc of ordering effects);
        matched patic (in pairs eg based on expertise, gender etc; adv: same as diff partic but indiv diff red; disadv: cannot be sure of perfect matching on all diff); \underline{Field Studies} - done in natural
        settings; aim to underst what usrs do naturally + how tech impacts them; important to: id oppor for new tech, determine design req, decide how best to introd new tech, eval tech in use;
        \underline{Data Collec + analy} - observ + interviews: notes, pictures, recordings, videp, logging; analyzes: categorised (by theory: grounded theory + act theory); \underline{Data pres} -
        aim is to show how prod bein approp + intergrated into their surroundings; typical presen forms include: vignettes (brief desc), excerpts, critical incid, patterns + narratives; \newline
        \textbf{Gathering Data} - setting goals (decide how to analyze data once collec); id partic (decide who to gather data from); relo w partic (clear + prof, informed consent when approp);
        triangulation (look at data from more than 1 persp, collect more than 1 type of data, eg qualitative from experim + qualitative from interviews); pilot studies (small trial of main study);
        \underline{Interviews} -
        unstruc: aren't directed by script, rich buts not replic; struc: are tightly scripted, often like ques, replic but may lack richness; semi-struc: guided by script but interesting issues
        can be explored in more depth + can provide good balance btw richness + replicability; focus groups: group interview; \underline{Interview Qs} - closed: have predetermined answer format; open: doesn't have;
        closed qs easier to analyse; avoid: long qs, compound sent (split them into 2), jargon + lang that interviewee may not underst, leading qs that make assump (eg why do you like...?), unconscious biases
        (eg gender stereotypes); \underline{Running Interview} - introd: introduce yourself, explain goals of interview, reassure about ethical issues, ask to record, present informed consent form; warm-up:
        make first qs easy + non-threatending; main body: present qs in logical order; cool-off period: include few easy qs to defuse tension at end; closure: than interviewee, signal end, eg. switch recorder off;
        \underline{Enriching Interview Process} - props: devices for prompting interviewee (eg. use prototype, scenario); ques: qs closed/open; closed qs easier to analyse + may be distrib
        + analysed by compl can be administered to large popul; disseminated by paper, email + web; sampling can be a prob when size of popul is unknown as is common online eval;
        \underline{Questionaire design} - impact of q can be influenced by q order; you may need diff ver of ques for iff popul; provide clear instruc on how to complete questionnaire;
        strike balance btw using white space + keeping ques compact; avoid very long ques; decide on whether phrases will all be +ve, all -ve or mixed; \underline{Q + Response format} -
        'yes' + 'no' checkboxes; checkboxes that offer many options; rating scales (likert scales (strongly disagr-strongly agr); semantic scales; 3,5,7+ points); open-ended responses;
        \underline{Encouraging a good response} - make sure purpose of study clear; promise anon; ensure ques well designed; offer short version for those who don't have time to complete long ques;
        if mailed $\to$ include stamped addr'd envelope; follow-up w emails, phone calls, letters; provide incentive; 40\% response rate good, 20\% acceptable; \underline{Adv of Online Ques} -
        relatively easy + quick to distrib; resources usually retrieved quickly; no copying + postage costs; data can be collec in database for analy; time req for data analysis red; errors
        corrected easily; \underline{Disadv of Online Ques} - sampling problematic if popul size unknown; preventing indiv from responding more than once can be prob; indiv have also been known to change
        qs in email ques; \underline{obser} - direct observation in field (struc frameworks, degree of partic (insider/outsider), ethnography); direct observation in controlled environ;
        indirect obser: tracking usrs' act (diaries, interac logging, videp + photog collected remotely by drones/other equip); \underline{Planning + conducting observation in field} - decide on how
        involved you'll be: passive observer to active participant; how to gain acceptance; how to handle sensitive topics (eg. culture, private spaces etc); how to collect data (what data to collect, what
        equip to use, when to stop observing); \underline{Ethnography} - philosophy w set of tchniques that include partic ovservation + intervieews; debte about diff btw paritic obser + ethnog; ethog'phers
        immerse themselves in culture they study; researcher's degree of partic can vary along scale from 'outside' to 'inside'; analysing vid + data logs can be time-consuming; collec of comments,
        incidents + artifects are made; co-oper of ppl being observed req; informants useful; data analy continuous; interpretivist technique; qs get refines as understing grows; reports
        usually contain examples; \underline{Online ethnog} - virtual, online, netnography; online + offline act; interac online differs from face-to-face; virtual worlds have persistance that physical worlds
        don't have; ethical considerations + presen of results diff; obser/materials that can be collec: act/job desc, rules + proced that govern part act, desc of act obser, recordings on convos,
        informal interviews, photos of artifacts; \underline{Obser in controlled environ} - direct obser (think-aloud techniques), indirect obser (tracking usrs' act, diaries, interac logs, web analytics);
        \underline{Web Analytics} - sys of tools + techniques for optimizing web usage by: measuring, collec, analysing + reporting web data; focus on no. of web visitors + pg views; \underline{Choosing + combining
        techniques} - depends on: focus of study, partic involved, nature of technique(s), resources available, time available; \newline \textbf{Data analy, interp + presen} \underline{Quant + qual} -
        quan analy: numerical methods to ascertain size, magnitude, amount (averages: mean, median, mode; \%s, graphs); qual analysis: expr nature of elem + is repr as themes, patterns, stories (patterns,
        themes, categorizing data); \underline{Tools for Data analy} - spreadsheet (simple to use, basic graphs); statistical packages (SPSS, R); qual data analy toolds (categorisation + theme-based analy, quan analy
        of text-based data); \underline{Theoretical Frameworks for Qual Analy} - basing data analy around theoretical frameworks provide further insight; frameworks: \underline{Grounded theory} -
        aims to derive theory from sys analy of data; based on categ approach ('coding'); 3 lvls of 'coding': open (id categories); axial (flesh out + link to subcate); selec (form theoretical scheme); researchers
        encouraged to draw on own theoretical backgrounds to inform analy; \underline{Distribution Cogn} - ppl environ + artefacts regarded as 1 cognitive sys; used for analysing collab work; focuses on info
        propagation + transformation; \underline{Resource Info Struc} - plans, goals, curr state, history (past actions), action-effect model (effect of actions on sys), affordances; \underline{Resource Config} -
        collec of info struc that can be defined for each step in an interac + which can be used to inform action; resources can be external in interf or rep in head of usr; inerac strategies link resource config
        to support decision making on actions; \underline{plan following} - plan, history + curr state; \underline{plan construc} - goal, affordances, action-effect + curr state; \underline{goal matching} - goal,
        affordances; \underline{history-based choice} - goal, affordances + history; resource alloc as part of UI design + id to support eval; \underline{Act Theory} - explains hmn behaviour in terms of practical
        act in world, provides framework that focuses analy around concep of 'act' + helps id tensions btw diff elem of sys; 2 key models: \underline{Indiv Model} - act $\leftrightarrow$ action $\leftrightarrow$ oper; motive $\leftrightarrow$ goal $\leftrightarrow$ cond;
        \underline{Rep findings} - only make claims that your data can support; best way to present findings depend on audience, purpose + data gathering + analy undertaken;
        graphical rep may be approp for pres; other techniques: rigorous notations (UML), using stories (scenarios), summarise findings; \newline \textbf{Introd Eval} \underline{Why, what, where, when to eval} - netural settings involving usrs (eg field studies, "in the wild" studies); settings not involving usrs (eg to predict,
        analyse + model aspects of interf anal); living labs: ppl's use of tech in everyday lives can be eval in living labs (eg aware home was embedded w complex network of sensors + audio/vid recording devices); Usability
        testing + field studies can compliment (field study to eval initial design ideas + get early feedback $\leftrightarrow$ make design changes $\leftrightarrow$ usab testing to check spec design features $\leftrightarrow$ field study tp see what happens
        when used in natural environ $\leftrightarrow$ make final design changes); \underline{Crowdsourcing} - what if we need large no. of partic; simil results to lab exp + lower cost; \underline{Eval methods} - observing + asking usrs;
        cont settings, natural settings; asking experts: natural settings + w/o usrs; testing: contr settings; modelling: w/o usrs; \underline{Partic's rights + getting consent} - partic need to be told why eval being done,
        what they will be asked to do + their rights; informed consent forms prov this info; design of inform consent form, eval, data analy + data storage methods typically approved by high authority (eg instit. review board);
        \underline{Things to consider when interpretting data} - reliability: does method prod same results on sep occasions; validity: does method measure what its intended to measure; ecol validity: does environ of eval distort
        results; biases: are there biases that distort results; scope: how generalised are results; \newline \textbf{Evaluation Framework} \underline{DECIDE framework} - determine goals (goal infl methods used for study),
        explore qs (qs help guide eval), choose eval methods (influ how data collected, analysed + presented); id practical issues (how to select usrs, find eval, select equip, stay on budget); decide how to deal w ethical issues
        (dev informed consent form); eval, analyse, interpret + present data; \newline \textbf{Eval: inspections, analytics + models} - inspec: experts use knowledge of usrs + tech to review SQ usab; expert critiques can be
        formal/informal; heuristic eval review guided by set of heuristics; wts involve stepping through pre-planned scenario noting potential prob; \underline{nielsen's heuristics}: visibility of sys status;
        match btw sys + real world; usr contr + freedom; consistency + standards; error prev; recog rather than recall; flex + effic of use; aesthetic + minimalist design; help usrs recog, diagn, recover from errors;
        help + docum.; \underline{Heuristics for sites} - focus on: clarity; min unnec complexity + cogn load; prov usrs w context; promo +ve + pleasurable usr exp; \underline{Stages of Heur eval} - briefing sess to tell experts
        what to do; eval period of 1-2 hrs where: expert works sep, take 1 pass to get feel for prod, take 2nd pass to focus on spec features; debriefing sess where experts work tgr to priori prob; \underline{Adv + disadv} -
        few ethical + prac issues to consider bc usrs not involved; can be hard + exp to find experts; best experts have knowledge of app domain + usrs; prob: important prob may get missed, many trivial prob
        often id'd, experts have biases; \underline{Cognitive wts} - focus on ease of learning; designer presents aspect of design + usage scenarios; expert told assump about usr popul, cotext of use, task
        details; 1+ experts walk through design proto w scenario; experts guided by 3q (1. will correct action be suffic evident to usr? 2. will usr notice that correct action avail? 3. will usr assoc + interp
        response from action correctly?); \underline{Pluristic wts} - var on cogn wt theme; perf by carefully manag team; panel of experts begin by working sep; managed disc that lends to agreed decis; approach leands itself
        to partic design; also other adap of basec cogn wts; \underline{Eval using analytics} - method for eval usr traffic through sys or part of sys; \underline{Predictive models} - provide way of eval prod/designs w/o
        directly involving usrs; less expendive than usr testing; usefulness limited to sys w predic tasks (eg telephone answering sys, mobiles, cell + smartphones); based on expert error-free behaviour; \underline{Fitts' Law} -
        predicts time to point at obj using device is func of distance from target obj + obj size; further away + smaller obj $\to$ longer time to locate it + point to it; important for eval sys where time to locate obj important
        (eg call + smartphones, handheld + mobile devices); \newline \textbf{hmn Factors + Security} \underline{hmn element} - HW, SW, liveware; 1 approach to sys to make them simple (red complex aids unders., not
        possible w addi of hmns as complex entities); hmn compo: often unreliable, not deterministic, vary enormously, issues w sys can be attributed to hmn error (dev vs usr distinc); \underline{Alloc of func} - comp good at:
        speed, following instruc; ppl good at: judgement calls, flex + adap; \underline{Sys complexity} - having hmn in sys loop adds complecity but shifts rep (implic for intergrity req on sys); many sys built on hmn strengths;
        \underline{hmns in sys} - producer, design team, cust, public, operator, maintainer, installer; \underline{hmn involvement} - hmn oper as part of sys; hmns who design, build + install sys; hmns who
        maintain sys; hmns certify sys (uses subj judgement calls); \underline{hmn error} - inevitable; slips: formulate right action but fail to exe action correctly; mistakes: fo not formulate right action; errors
        occue despite exp; \underline{Types of hmn error} - skill-based: usually errors of inattention/misplaced attention; rule-based: picked inapprop rule misdiag state of sys, defic rules; knowledge-based: incomplete/inacc
        understing of sys, overconf, cogn strain; sys design affects error-susceptibility (bad design, poor usability $\to$ HCI isues); \underline{Dangers of Autom} - autom often addr skill + rule-based tasks (complec
        knowledge-bsed tasks left for hmn oper; under stress, hmns ill-suited to knowledge-based tasks); \underline{Implic of autom} - can hinder underst + mental modelling (dec sys vidib + inc complex, oper don't get
        hands-on contr exp); \underline{hmn involvement unavoidable} - any involvement prov opport of errors; build sys that can cope w hmn error (fault tolerance sys. focus on recovery); 'undo' (trial + error, "time
        travel" for sys state); \underline{hmns in eval} - part of proc of verif + eval sys; verif: proc of determining whether output of lifecycle phase fulfils req soec by orev phase; valida: proc of confirming spec
        of phase or fin sys is approp + consistent w cust req; testing: proc used to verif/valid sys or its compon; \underline{V dev model w test planning}; \underline{Sec as non-req} - depen: reliab, availab, safety, sec;
        \underline{Sec considerations} - sec sys attr that reflability sys to protect itself from external attacks (attacks may be accid/delib); influ of internet (inc opport for attack, inc info dist); poor interf design
        can inc sys's vuln to other attacks; \underline{Sec Aspects} - confid: prev of unauthorised disclosure of info; integ: prev of unauthorised modif of info; avail: prev of unauth withholding of info of resources;
        \underline{Sec of critical sys} - sec most inmportant of depend attrib (military sus, elec commerce, sys involved in proc + interchange of confid info); req for high lvl of sec; reliab + avail may cause inconven but
        sec compromise whole sys environ; \underline{Points of attack} - sys consist of hosts (servers, clients), networks, apps, sys admin, usrs + attackers; points of attack at host + commun links: ppl - social eng;
        server - malic input for buffer overflows; SW - upgrades + viruses; data - config data; network - "spoofing" to access traffic; \underline{Categ of attack} - intercep (confid), nodif (integ), interrup (avail);
        fabrication (auth); DoS; correup of prog/data; disclosure of conf info; \underline{Alienation} - comp can contrib to feelings of alien in society;
        \underline{\textbf{Anxiety}}: ease of use (natural UI, forgiving sys, usr control);
        \textbf{Alienation}: socially active designers, comms support sys can indicate emotions (emoticons), multimedia sys incl videp \& voice;
        \textbf{Impotency}: when simple task put incontact with comp may feel powerless, frustrated;
        \textbf{Complexity \& Speed}: comp speed up \& add cmplx to life, should try to simplify \& slow down some;
        \textbf{Org \& Societal Dep}: dep on sys make it easier to bring down, need robust fault tolerant design;
        \textbf{Unemployment \& Displacement}: responsibility of org, offer retraining \& other jobs;
        \textbf{Value hmn Diversity}: flex, adap intf, neutral, non-biased interaction;
        \textbf{Ethical Concerns}: accessibility, privacy, cyberbully, accountability, accuracy, online predator, intellectual property, usr testing (perf anxiety, feel like intelligence test, compare self \& compete with others, feel stupid in front of observer);
        \underline{\textbf{Formative vs Summative}}: F used throughout, S at end, F for improvement, S for grades/check of goals met at end;

    \end{multicols}
\end{document}
