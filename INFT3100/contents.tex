\documentclass[a4paper]{article}
    \usepackage[margin=0.25in]{geometry}
    \usepackage{multicol}

    \begin{document}
    \begin{multicols}{2}
        \scriptsize
        \noindent\textbf{Project}: complex, non-routine, one-time effort limited by time, budget, resources, perf specs designed to meet customer needs\\
        \textbf{Project characteristics}: established objective, defined life span with beginning \& end, requires participation across the org, typically involves doing sth never done before, has specific time, cost, perf reqs\\
        \textbf{Project life cycle}:\\
        \textit{Defining}: goals, specs, tasks, responsibilities\\
        \textit{Planning}: schedules, budgets, resources, risks, staffing\\
        \textit{Executing}: status reports, changes, quality, forecasts\\
        \textit{Closing}: train customer, transfer docs, release resources, evaluation, lessons learned\\
        \textbf{Factors leading to increased use of PM}: compression of product life cycle, knowledge explosion, triple bottom line (planet, people, profit), corporate downsizing, increased customer focus, small projects represent big problems\\
        \textbf{Project Manager}: manages temporary, non-repetitive activities, frequently acts independently of formal org, marshals resources for project, linked directly to customer interface, provides direction, coordination \& integration to the project team, responsible for performance \& success of the project, must induce right people at right time to address the right issues and make right decisions\\
        \textbf{PM Technical Aspects}: scope, WBS, schedules, resource allocation, baseline budgets, status reports\\
        \textbf{PM Sociocultural Aspects}: leadership, problem solving, teamwork, negotiation, politics, customer expectations\\
        \textbf{Integrated management of projects}: strategic alignment, portfolio management, PM, with org culture env wrapped around\\
        \textbf{Integration of projects with org strategy}: use of selection criteria to ensure strategic alignment and project priorities (effective use of org resources); selection process that is systematic, open, consistent \& balanced; all selected projects become part of portfolio that balances risk for org; portfolio management ensures only most valuable projects approved \& managed across entire org; value of project not only ROI but also strategic fit \& best use of org resources\\
        \textbf{Integrative PM approach benefits}: provide senior management with overview of all PM activities, big picure of how org resources used, risk assessment of project portfolio, rough metric of org's improvement in managing projects relative to others in industry, linkages of senior management with actual project execution management\\
        \textbf{Portfolio Management Functions}: oversee project selection, monitor aggregate resource levels \& skills, encourage use of best practices, balance projects in portfolio in order to represent risk level appropriate to the organisation, improve communication among all stakeholders, create total ord perspective that goes beyond silo thinking, improve overall management of projects over time\\
        \textbf{Program}: a series of coordinated, related, multiple projects that continue over an extended time and are intended to achieve a goal\\
    \end{multicols}
    \end{document}
