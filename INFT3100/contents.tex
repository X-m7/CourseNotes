\documentclass[a4paper]{article}
    \usepackage[margin=0in]{geometry}
    \usepackage{multicol}

    \begin{document}
    \begin{multicols}{2}
        \tiny
        \noindent\underline{\textbf{Project}}: complex, non-routine, one-time effort limited by time, budget, resources, perf specs designed to meet customer needs\\
        \textbf{Project characteristics}: established objective, defined life span with beginning \& end, requires participation across the org, typically involves doing sth never done before, has specific time, cost, perf reqs\\
        \textbf{Project life cycle}:\\
        \textit{Defining}: goals, specs, tasks, responsibilities\\
        \textit{Planning}: schedules, budgets, resources, risks, staffing\\
        \textit{Executing}: status reports, changes, quality, forecasts\\
        \textit{Closing}: train customer, transfer docs, release resources, evaluation, lessons learned\\
        \textbf{Factors leading to increased use of PM}: compression of product life cycle, knowledge explosion, triple bottom line (planet, people, profit), corporate downsizing, increased customer focus, small projects represent big problems\\
        \textbf{Project Manager}: manages temporary, non-repetitive activities, frequently acts independently of formal org, marshals resources for project, linked directly to customer interface, provides direction, coordination \& integration to the project team, responsible for performance \& success of the project, must induce right people at right time to address the right issues and make right decisions\\
        \textbf{PM Technical Aspects}: scope, WBS, schedules, resource allocation, baseline budgets, status reports\\
        \textbf{PM Sociocultural Aspects}: leadership, problem solving, teamwork, negotiation, politics, customer expectations\\
        \textbf{Integrated management of projects}: strategic alignment, portfolio management, PM, with org culture env wrapped around\\
        \textbf{Integration of projects with org strategy}: use of selection criteria to ensure strategic alignment and project priorities (effective use of org resources); selection process that is systematic, open, consistent \& balanced; all selected projects become part of portfolio that balances risk for org; portfolio management ensures only most valuable projects approved \& managed across entire org; value of project not only ROI but also strategic fit \& best use of org resources\\
        \textbf{Integrative PM approach benefits}: provide senior management with overview of all PM activities, big picure of how org resources used, risk assessment of project portfolio, rough metric of org's improvement in managing projects relative to others in industry, linkages of senior management with actual project execution management\\
        \textbf{Portfolio Management Functions}: oversee project selection, monitor aggregate resource levels \& skills, encourage use of best practices, balance projects in portfolio in order to represent risk level appropriate to the organisation, improve communication among all stakeholders, create total ord perspective that goes beyond silo thinking, improve overall management of projects over time\\
        \textbf{Program}: a series of coordinated, related, multiple projects that continue over an extended time and are intended to achieve a goal\\
        \underline{\textbf{Traditional PM}}: focus on thorough, upfront planning of entire project, requires high degree of predictability to be effective\\
        \textbf{Agile}: relies on incremental, iterative dev cycles to complete less predictable projects, ideal for exploratory projects in which requirements need to be discovered and new tech tested (uncertain ablut how long, what is required, allows change in reqs), focus on active collaboration between project tem \& customer reps\\
        \begin{tabular}{l l}
            \textbf{Traditional} & \textbf{Agile}\\
            Design up front & Continuous design\\
            Fixed scope & Flexible\\
            Deliverables & Features/reqs\\
            Freeze design as early as possible & as late as possible\\
            Low uncertainty & high\\
            Avoid change & embrace\\
            Low customer interaction & high\\
            Conventional project teams & self-organised\\
        \end{tabular}\\
        \textbf{Agile Details}: use iterations to develop workable product that satisfies the customer and other key stakeholders, stakeholders \& customers review progress \& re-evaluate priorities to ensure alignment with customer needs \& company goals, adjustments are made \& a different iterative cycle begins that subsumes the work of the previous iterations \& adds new capabilities to the evolving product\\
        \textbf{Agile Advantages}: useful in developing critical breakthrough tech or defining essential features; continuous integration, verification \& validation of the evolving product; frequent demonstration of progress to increase likelihood that end product will satisfy customer needs; early detection of defects \& problems\\
        \textbf{Agile Limitations}: does not satisfy top management's need for budget, scope \& schedule control; self-organisation \& close collaboration principles can be incompatible with corporate cultures; appears to work best on small project with 5--9 people, requires active customer involvement \& cooperation\\
        \textbf{Agile Principles}: focus on customer value, iterative \& incremental delivery, experimentation \& adaptation, self-organisation, continuous improvement\\
        \textbf{Project uncertainty dimensions}: scope \& tech
        \textbf{Scrum}: holistic (interconnected emphasis) approach for use by cross-functional team collaborating to develop new product, defines product features as deliverables \& prioritises them by perceived highest value to the customer, re-evaluates priorities after each iteration/sprint to produce fulle functional features, phases: analysis, design, build, test\\
        \textbf{Scrum roles \& responsibilities}:\\
        \textit{Product owner}: acts on behalf of customer to represent interests, responsible for product backlog priorities \& process selection\\
        \textit{Development team}: 5--9 people with cross-functional skillsets responsible for delivering product, sets own goals, organises itself, makes decisions\\
        \textit{Scrum master}: facilitates scrum process and resolves impediments at the team \& org level by acting as buffer between team \& outside interference\\
        \textbf{Scrum practices}:\\
        \textit{Sprint}: time-controlled mini-project that implements specific portion of a system, 30 day time box with specific goals \& deliverables, frozen scope defined from sprint backlog\\
        \textit{Daily Scrum}: daily meeting of all team members to report progress (15 min max), also called standup\\
        \textit{Sprint final half-day review meeting}: review \& identify changes needed for following sprints\\
        \textbf{Sprint meetings}: sprint planning, daily scrum, sprint review, sprint retrospective\\
        \textbf{Product backlog}: customer's prioritised list of desired key features for the completed project, can only be changed by product owner\\
        \textbf{Sprint backlog}: amount of work team commits to complete during the next sprint, developed \& controlled by team\\
        \textbf{Scaling}: using several teams to work on different features of large scale project at same time\\
        \textbf{Staging}: upfront planning to manage interdependencies of the different features to develop, involves developing protocols \& defining roles to coordinate efforts \& assure compatibility \& harmony\\
    \end{multicols}
    \end{document}
