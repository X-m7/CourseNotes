\documentclass[a4paper]{article}
    \usepackage[margin=0in]{geometry}
    \usepackage{multicol}

    \begin{document}
    \begin{multicols}{2}
        \tiny
        \noindent\underline{\textbf{Project}}: complex, non-routine, one-time effort limited by time, budget, resources, perf specs designed to meet customer needs\\
        \textbf{Project characteristics}: established objective, defined life span with beginning \& end, requires participation across the org, typically involves doing sth never done before, has specific time, cost, perf reqs\\
        \textbf{Project life cycle}:\\
        \textit{Defining}: goals, specs, tasks, responsibilities\\
        \textit{Planning}: schedules, budgets, resources, risks, staffing\\
        \textit{Executing}: status reports, changes, quality, forecasts\\
        \textit{Closing}: train customer, transfer docs, release resources, evaluation, lessons learned\\
        \textbf{Factors leading to increased use of PM}: compression of product life cycle, knowledge explosion, triple bottom line (planet, people, profit), corporate downsizing, increased customer focus, small projects represent big problems\\
        \textbf{Project Manager}: manages temporary, non-repetitive activities, frequently acts independently of formal org, marshals resources for project, linked directly to customer interface, provides direction, coordination \& integration to the project team, responsible for performance \& success of the project, must induce right people at right time to address the right issues and make right decisions\\
        \textbf{PM Technical Aspects}: scope, WBS, schedules, resource allocation, baseline budgets, status reports\\
        \textbf{PM Sociocultural Aspects}: leadership, problem solving, teamwork, negotiation, politics, customer expectations\\
        \textbf{Integrated management of projects}: strategic alignment, portfolio management, PM, with org culture env wrapped around\\
        \textbf{Integration of projects with org strategy}: use of selection criteria to ensure strategic alignment and project priorities (effective use of org resources); selection process that is systematic, open, consistent \& balanced; all selected projects become part of portfolio that balances risk for org; portfolio management ensures only most valuable projects approved \& managed across entire org; value of project not only ROI but also strategic fit \& best use of org resources\\
        \textbf{Integrative PM approach benefits}: provide senior management with overview of all PM activities, big picure of how org resources used, risk assessment of project portfolio, rough metric of org's improvement in managing projects relative to others in industry, linkages of senior management with actual project execution management\\
        \textbf{Portfolio Management Functions}: oversee project selection, monitor aggregate resource levels \& skills, encourage use of best practices, balance projects in portfolio in order to represent risk level appropriate to the organisation, improve communication among all stakeholders, create total ord perspective that goes beyond silo thinking, improve overall management of projects over time\\
        \textbf{Program}: a series of coordinated, related, multiple projects that continue over an extended time and are intended to achieve a goal\\
        \underline{\textbf{Traditional PM}}: focus on thorough, upfront planning of entire project, requires high degree of predictability to be effective\\
        \textbf{Agile}: relies on incremental, iterative dev cycles to complete less predictable projects, ideal for exploratory projects in which requirements need to be discovered and new tech tested (uncertain ablut how long, what is required, allows change in reqs), focus on active collaboration between project tem \& customer reps\\
        \begin{tabular}{l l}
            \textbf{Traditional} & \textbf{Agile}\\
            Design up front & Continuous design\\
            Fixed scope & Flexible\\
            Deliverables & Features/reqs\\
            Freeze design as early as possible & as late as possible\\
            Low uncertainty & high\\
            Avoid change & embrace\\
            Low customer interaction & high\\
            Conventional project teams & self-organised\\
        \end{tabular}\\
        \textbf{Agile Details}: use iterations to develop workable product that satisfies the customer and other key stakeholders, stakeholders \& customers review progress \& re-evaluate priorities to ensure alignment with customer needs \& company goals, adjustments are made \& a different iterative cycle begins that subsumes the work of the previous iterations \& adds new capabilities to the evolving product\\
        \textbf{Agile Advantages}: useful in developing critical breakthrough tech or defining essential features; continuous integration, verification \& validation of the evolving product; frequent demonstration of progress to increase likelihood that end product will satisfy customer needs; early detection of defects \& problems\\
        \textbf{Agile Limitations}: does not satisfy top management's need for budget, scope \& schedule control; self-organisation \& close collaboration principles can be incompatible with corporate cultures; appears to work best on small project with 5--9 people, requires active customer involvement \& cooperation\\
        \textbf{Agile Principles}: focus on customer value, iterative \& incremental delivery, experimentation \& adaptation, self-organisation, continuous improvement\\
        \textbf{Project uncertainty dimensions}: scope \& tech
        \textbf{Scrum}: holistic (interconnected emphasis) approach for use by cross-functional team collaborating to develop new product, defines product features as deliverables \& prioritises them by perceived highest value to the customer, re-evaluates priorities after each iteration/sprint to produce fully functional features, phases: analysis, design, build, test\\
        \textbf{Scrum roles \& responsibilities}:\\
        \textit{Product owner}: acts on behalf of customer to represent interests, responsible for product backlog priorities \& process selection\\
        \textit{Development team}: 5--9 people with cross-functional skillsets responsible for delivering product, sets own goals, organises itself, makes decisions\\
        \textit{Scrum master}: facilitates scrum process and resolves impediments at the team \& org level by acting as buffer between team \& outside interference\\
        \textbf{Scrum practices}:\\
        \textit{Sprint}: time-controlled mini-project that implements specific portion of a system, 30 day time box with specific goals \& deliverables, frozen scope defined from sprint backlog\\
        \textit{Daily Scrum}: daily meeting of all team members to report progress (15 min max), also called standup\\
        \textit{Sprint final half-day review meeting}: review \& identify changes needed for following sprints\\
        \textbf{Sprint meetings}: sprint planning, daily scrum, sprint review, sprint retrospective\\
        \textbf{Product backlog}: customer's prioritised list of desired key features for the completed project, can only be changed by product owner\\
        \textbf{Sprint backlog}: amount of work team commits to complete during the next sprint, developed \& controlled by team\\
        \textbf{Scaling}: using several teams to work on different features of large scale project at same time\\
        \textbf{Staging}: upfront planning to manage interdependencies of the different features to develop, involves developing protocols \& defining roles to coordinate efforts \& assure compatibility \& harmony\\
        \underline{\textbf{Strategy Importance}}: Project managers must respond to changes to organisation mission and strategy appropriately, if understand strategy can become effective advocates of projects aligned with firm's mission\\
        \textbf{Mistakes by not understanding role of projects in accomplishing strategy}: focus on problems/solutions with low strategic priority, focus on immediate customer rather than whole marketplace \& value chain, over-emphasising tech that results in projects that pursue exotic tech that does not fit strategy or customer need, trying to solve customer issues with product/service rather than focusing on 20\% with 80\% of value (\textbf{Pareto's law}), engaging in never-eding search of perfection that only team really cares about\\
        \textbf{Strategic management}: requires every project to be clearly linked to strategy; provides theme \& focus of organisational future direction (responding to changes in external env --- env scanning, allocating scarce resources of firm to improve competitive position --- internal responses to new programs); requires strong links among mission, goals, objectives, strategy, impl\\
        \textbf{Strategic management activities}:\\
        \textit{Review \& define org mission}: identify \& communicate purpose of org to stakeholders, identify scope of org in terms of product/service, provides focus for decision making, used for eval org perf\\
        \textit{Set long-range goals \& objectives}: translate mission to specific, concrete \& measurable terms; sets targets for all levels of org in a cascaded manner; where is org headed and when it will get there; focus managers on where org should move to\\
        \textit{Analyse \& formulate strategies to reach objectives}: focus on what needs to be done to reach objectives, relaistic view of past \& current position, SWOT analysis, alternatives generated \& assessed, strategy formulation \& assignation\\
        \textit{Implement strategies through projects}: focus on how strategies will be realised with resources, maintain link between strategy (what) \& impl (how), requires resource allocation, action \& completion of tasks, prioritisation\\
        \textbf{SMART objectives}: Specific, Measurable (indicators of progress), Assignable (to one person for completion), Realistic (what can realistically be done with avail resources), Time related (state when objective can be achieved)\\
        \textbf{SWOT analysis}: internal (strengths, weaknesses) \& external (opportunities, threats) analysis\\
        \textbf{Scenario planning}: longer term, steps: clarifying core business \& assessing drivers of change in industry env, dev potential scenatios \& assess impact of STEEP factors, dev potential contingency strategies \& best future strategic options, identifying early indicators \& establishing triggers for strategic action\\
        \textbf{STEEP factors}: social, tech, env, economic, political\\
        \textbf{Project portfolio management benefits}: build discipline to project selection process, link project selection to strategic metrics, prioritise project proposal across common set of criteria rather than politics/emotion, allocate resources to projects that align with strategic direction, balance risk across all projects, justifies stopping projects that don't support strategy, improves comms \& supports agreement on project goals\\
        \textbf{Project portfolio management problems}:\\
        \textit{Implementation gap}: lack of understanding \& consensus on strategy among to management \& middle-level (functional) managers who independently implement strategy\\
        \textit{Org politics}: project selection based on persuasiveness \& power of people advocating projects\\
        \textit{Resource conflicts \& multitasking}: Multiproject env creates interdependency relationships of shared resources which results in starting, stopping \& restarting of projects\\
        \textbf{Project portfolio sys design}: classification of project, selection criteria depending upon classification, sources of proposals, evaluating proposals, ranking proposals, managing portfolio of projects\\
        \textbf{Project types}: compliance (must-do, incl emergency, meet regulations, usually have penalties if not impl), strategic (directly support long-run mission, increase revenue/market share, ex: new products, R\&D), operational (support current ops, improve perf, reduce product cost, improve efficiency of delivery sys, ex: upgrade building green rating)\\
        \textbf{Financial Selection Criteria}: payback, NPV, IRR (internal rate of return, inverse of payback)\\
        \textbf{Payback model}: measures time project takes to recover investment; uses more desirable, shorter paybacks; emphasises cash flows (key factor in business)\\
        \textbf{Payback limitations}: ignores time value of money, assumes casf inflow only for investment period, does not consider profitability\\
        \textbf{Net Present Value}: $I_0 + \sum\nolimits_{t=1}^n\frac{F_t}{{(1 + k)}^t}$, $I_0$ is initial investment (negative), $F_t$ is net cash inflow for period $t$, $k$ is required rate of return, want positive\\
        \textbf{Non-financial strategic criteria}: capture larger market share, make it difficult for competitors to enter the market, develop enabler product which by interduction will increase sales in more profitable products, develop core tech to be used in next-gen products, reduce dependency on unreliable suppliers, prevent government intervention \& regulation\\
        \textbf{Multicriteria selection models}:\\
        \textit{Checklist model}: use list of questions to review potential projects \& to determine accept/reject, fails to answer relative importance/value of potential project \& doesn't allow for comparison with others\\
        \textit{Multiweighted scoring model}: use several weighted qualitative and/or quantitative selection criteria to evaluate project proposals, can use for comparison\\
        \textbf{Selection model advantages}: bring projects to closer alignment with org strategic goals, reduce number of wasteful projects, help identify proper goals for projects, help everyone involved understand how \& why project is selected\\
        \textbf{Project relativity matrix}: 2 dimensions (technical feasibility, NPV), white elephant (low, low, showed promise at one time but are no longer viable), oyster (low, high, technological breakthroughs with high commercial payoffs), bread-and-butter (high, low, evolutionary improvements to current products \& services), pearl (high, high, revolutionary commercial opportunities using proven tech advances)\\
        \underline{\textbf{Challenges to organising projects}}: need to balance needs of project with org, uniqueness \& short duration of projects relative to ongoing longer term org activities, multidisciplinary \& cross-functional nature of projects creates authority \& responsibility dilemmas\\
        \textbf{Functional org}: different segments of project delegated to functional units, coordination maintained through normal management channels, used when interest of 1 functional area dominates project or has dominant interest in project success\\
        \textbf{Functional +}: no structural change, flexibility, in-depth expertise, easy post-project transition\\
        \textbf{Functional -}: lack of focus, poor integration, slow, lack of ownership\\
        \textbf{Dedicated project teams}: teams operate as separate units under leadership of full-time project manager, in projectised org where projects are dominant form of business functional depts are responsible for providing support to teams\\
        \textbf{Dedicated +}: simple, fast, cohesive, cross-functional integration\\
        \textbf{Dedicated -}: expensive, internal strife, limited tech expertise, difficult post-project transition\\
        \textbf{Hybrid/Matrix}: overlaid on normal functional structure, 2 chains of command (functional \& project), project participants report simultaneously to both functional \& project managers, optimise use of resource (allows pariticipation on multiple projects while performing normal functional duties)\\
        \textbf{Matrix +}: efficient, strong project focus, flexible, easy post-project transition\\
        \textbf{Matrix -}: dysfunctional conflict, infighting, slow, stressful\\
        \textbf{Weak matrix}: authority of functional manager predominates, project manager has indirect authority\\
        \textbf{Balanced matrix}: the project manager sets overall plan \& the functional manager determines how work is to be done\\
        \textbf{Strong matrix}: project manager has broader control, functional departments act as subcontractors to project\\
        \textbf{Matrix division of responsibilities}:\\
        \textit{Project manager}: what has to be done, when should the task be done, how much money is available to do the task, how well has the total project been done\\
        \textit{Functional manager}: how will it be done, how will project involvement impact normal functional activities, how well has the functiona input been integrated\\
        \textit{Negotiated issues}: who will do the task, where will the task be done, why will the task be done, is the task satisfactorily completed\\
        \textbf{Choosing the appropriate project management structure}:\\
        \textit{Organisational considerations}: how important is the project to the firm's success, what percentage of core work involves projects, what level od resources (human \& physical) are available\\
        \textit{Project considerations}: size of project, strategic importance, novelty \& need for innovation, need for integration (number of depts involved), environmental complexity (number of external interfaces), budget \& time constraints, stability of resource reqs\\
        \textbf{Org culture}: system of shared norms, beliefs, values \& assumptions that bind people together, thereby creating shared meanings; personality of org that sets it apart from other orgs\\
        \textbf{Org culture benefits}: provides sense of identity to members, helps legitimise management system of org, clarifies \& reinforces standards of behaviour, helps create social order\\
        \textbf{Diagnosing org culture}: study physical characteristics (architecture, office layout, decor, attire), read about org (annual reports, internal newsletters, vision statements), observe how people interact within org (pace, lang, meetings, issues discussed, decision-making style, comm patterns, rituals), interpret stories \& folklore surrounding org (anecdotes, heroines, heroes, villains)\\
        \textbf{Org culture dimensions}: member identity (job, org), team emphasis (individual, group), management focus (task, people), unit integration (independent, interdependent), control (loose, tight), risk tolderance (low, high), reward criteria (performance, other), conflict tolerance (low, high), means-ends orientation (means, ends), open-system focus (internal, external, degree to which org monitors \& responds to changes in external env)\\
        \textbf{Culture challenges for structuring projects}: interacting with culture \& subcultures of parent org, interacting with project clients or customer orgs, interacting with other orgs connected to project\\
        \textbf{Mechanisms for sustaining org culture}: formal statement of principles, top management behaviour, reactions to org crises, allocation of rewards \& status, rituals, stories, symbols\\
        \underline{\textbf{Defining the project}}: defining project scope, establishing project priorities, creating WBS, integrating WBS with org, coding WBS for information sys\\
        \textbf{Project scope}: definition of end result or mission of project --- a product/service for clinet/customer --- in specific, tangible \& measurable terms\\
        \textbf{Scope statement}: statement of work (SOW)\\
        \textbf{Scope statement purpose}: clearly define deliverables for end user, focus project on successful completion of its goals, to be used by project owner \& participants as planning tool \& measuring project success\\
        \textbf{Project scope checklist}: project objective; deliverables; milestones; technical reqs; limits \& exclusions; reviews with customer\\
        \textbf{Project charter}: can contain expanded version of scope statement, document authorising project manager to initiate \& lead project\\
        \textbf{Scope creep}: tendency for project scope to expand over time due to changing requirements, specs, priorities\\
        \textbf{Priority matrix}: budget/cost, schedule/time, performance/scope, constrain, enhance (optimise), accept\\
        \textbf{Work Breakdown Structure}: hierarchical outline (map) that identifies products \& work elements involved in project, defines relationship of final deliverable to subdeliverables \& in turn their relationships to work packages, best suited for design \& build projects that have tangible outcomes rather than process-oriented projects\\
        \textbf{WBS Hierarchy}: project, deliverable, sub-deliverable, lowest sub-deliverable (lowest management responsibility level), cost account (group of work packages for monitoring progress \& responsibility), work package\\
        \textbf{WBS benefits fro project manager}: facilitates evaluation of cost, time \& technical perf of org on project; provides management with info appropriate to each org level; helps in dev of OBS, which assigns project responsibilities to org units \& individuals; help manage plan, schedule \& budget; define comm channels \& assists in coordinating various project elements\\
        \textbf{Work Package}: defines work (what), identifies time to complete, time-phased budget to complete (cost), resources needed to complete (how much), person responsible for units of work, monitoring points/milestones for measuring success (how well)\\
        \textbf{Org Breakdown Structure}: how company organised to discharge work responsibility for project\\
        \textbf{OBS details}: provides framework to summarise org work unit perf, identifies org units responsible for work packages, ties org units to cost control accounts\\
        \textbf{Intersection of WBS \& OBS}: project control point/cost account\\
        \textbf{WBS coding system}: defines levels \& elements of WBS, org elements, work packages, budget \& cost info, allows reports to be consolidated at any level in org structure\\
        \textbf{Responsibility Matrix}: linear responsibility chart, summarises tasks to be accomplished \& who is responsible for what on the project\\
        \textbf{RM details}: list project activities \& participants, clarifies critical interfaces between units \& individuals that need coordination, provides means for all participants to view responsibilities \& agree on assignments, clarifies extent/type of authority that can be exercised by each participant\\
        \textbf{Project communication plan}: what info needs to be collected \& when, who will receive info, what methods will be used to gather \& store info, what are limits on who as access to certain kinds of info, when will info be communicated, how will it be communicated\\
        \textbf{Comm plan steps}: stakeholder analysis, info needs, soruces of info, dissemination modes, responsibility \& timing\\
        \textbf{Information needs}: project status reports, deliverable issues, changes in scope, team status meetings, getting decisions, accepted request changes, action items, milestone reports\\
        \underline{\textbf{Estimating}}: process of forecastine/approximating time \& cost of completing project deliverables, task of balancing exepctations of stakeholders\& need for control while project is implemented\\
        \textbf{Estimating importance}: support good decisions, schedule work, determine how long project should take \& cost, determine whether project worth doing, develop cash flow needs, determine how well the project is progressing, develop time=phased budgets \& establish project baseline\\
        \textbf{Estimation accuracy factors}: planning horizon, project duration, people, project structure \& org, padding estimates, org culture, other non-project factors\\
        \textbf{Estimating guidelines}: have people familiar with tasks make estimate; use several people to make estimates; base estimates on normal conditions, efficient methods \& normal level of resources; use consistent time units; treat each task as independent; don't make allowances for contingencies, adding risk assessment helps avoid surprises to stakeholders\\
        \textbf{Top-down estimates}: derived from someone who uses experience and/or info to determine the project duration \& total cost, are made by top managers who have little knowledge of the processes used to complete the project, time \& costs are not considered, grouping tasks may lead to omissions \& unrealistic times \& costs, accuracy -20\% to +60\%, cost 0.1--0.3\%\\
        \textbf{Conditions for top-down}: strategic decision making; high uncertainty; internal, small project, unstable scope\\
        \textbf{Top-down intended use}: feasibility/conceptual phase, rough time/cost estimate, fund reqs, resource capacity planning\\
        \textbf{Bottom-up approach}: can serve as a check on cost elements in WBS by rolling up work packages \& associated cost accounts to major deliverables at work package level, more accurate but takes more time, accuracy level may not be required for some projects, accuracy -10\% to +30\%, cost 0.3--1\%\\
        \textbf{Conditions for bottom-up}: cost \& time important, fixed-price contract, customer wants details\\
        \textbf{Bottom-up intended use}: budgeting, scheduling, resource reqs, fund timing\\
        \textbf{Preferred estimating approach}: rough top-down estimates, dev WBS/OBS, make bottom-up estimates, dev schedules \& budgets, reconcile diffs between top-down \& bottom-up estimates\\
        \textbf{Top-down approaches}:\\
        \textit{Consensus}: use xp of senior and/or mid managers to estimate total project duration \& cost; typically involve meeting where experts discuss, argue \& ultimately reach decision for best guessestimate\\
        \textit{Delphi}: about likelihood that certain events will occur, ask experts, then return summary of opinions (anon), encourage to reconsider/change based on others' opinions, repeat 2--3x, median will move toward `correct' estimate, avoid ego, domineerring, bandwagon, halo effect, no need for physical contact\\
        \textit{Ratio}: use cost/time per area/capacity size/features/complexity\\
        \textit{Apportion}: extension of ratio, use if projects closely follow past projects in features \& cost, pay contractor by completion of parts or split costs based on deliverables in WBS (each has percent allocated)\\
        \textit{Function point}: for software \& system projects, take several elements (input, output, inquiries, files, interfaces), rate complexity, multiply number of each with complexity, total is estimate\\
        \textit{Learning curve}: take number of units \& improvement rate\\
        \textbf{Bottom-up approaches}:\\
        \textit{Template}: start wirh standard task cost/time estimates then adjust specifics\\
        \textit{Parametric applied to specific tasks}: need to do X work, 1 person can do Y work in Z time\\
        \textit{Range estimates for work packages}: low, average, high for each, useful if work packages have significant uncertainty\\
        \textbf{Phase estimating}: hybrid top-down \& bottom-up, macro long-term (rest of project) \& micro short-term (current phase, need, specs, design, produce, deliver), for projects with high uncertainty, customers may be able to change features \& re-evaluate project at each stage, but customers \& clients typically want form estimates of time \& overall cost up front\\
        \textbf{Level of detail}: varies in WBS with project complexity, each management level can focus on what they need; excessive detail is costly, fosters focus on departmental outcpmes \& create unproductive paperwork but insufficient detail is also costly, fosters lack of focus on goals \& leads to wasted effort on non-essential activities\\
        \textbf{Cost types}:\\
        \textit{Direct}: clearly chargeable to specific work package, ex: labour, materials, equipment\\
        \textit{Direct (project) overhead}: directly tied to identifiable deliverable/work package, ex: salary, rents, supplies, specialised machinery\\
        \textit{General \& administrative overhead}: indirectly linked to specific package apportioned to project, carried for project duration, ex: ads, accounting, senior management\\
        \textbf{Cost views}: committed, scheduled budget, actual cost\\
        \textbf{Adjusting estimates}: done for specific activities as risks, resources \& situation particulars become more actively defined, mitigate risks by recognising mistakes can occur (ex: include independent testers to check design), allowing for difficult conditions in contracts affecting scope (if excessive ground water adjust foundation estimates)\\
        \textbf{Reasons for adjusting estimates}: interaction costs hidden in estimates, normal conditions do not apply, things go wrong in projects, changes in project scope \& plans\\
        \textbf{Estimating DB}: estimated \& actuals on labour, costs, equipment, benchmarking ratios, code of accounts for various project types\\
        \underline{\textbf{Project network}}: flow chart that graphically depicts sequence, interdependencies, start \& finish times of project job plan of activities\\
        \textbf{Critical path}: longest activity paths through network that allows for completion of all activities; shortest expected time in which entire project can be completed; 0 slack, also consider deps caused by resource constraints\\
        \textbf{Project network benefits}: provides basis for scheduling labour \& equipment, enhance comms among project participants, provides estimate of project duration, provides basis for budgeting cash flow, highlights `critical' activities that cannot be delayed, highlights activities that can be compressed to meet deadline, help managers get \& stay on plan\\
        \textbf{Late finish}: latest activity can finish \& not delay following activity, LS + DUR\\
        \textbf{Late start}: latest activity can start \& not delay following activity, LF --- DUR\\
        \textbf{Early finish}: earliest an activity can finish if all preceding activities are finished by early finish, ES + DUR\\
        \textbf{Early start}: earliest an activity can start, largest early finish of all predecessors, EF ---  DUR\\
        \textbf{Activity}: project element that requires time\\
        \textbf{Merge}: activity with 2 or more preceding activities on which it depends\\
        \textbf{Parallel/concurrent}: can occur independently \& if desired not at the same time\\
        \textbf{Path}: sequence of connected, dependent activities\\
        \textbf{Event}: point in time when activity started/completed, does not consume time\\
        \textbf{Burst}: activity with more than one activity immediately following it\\
        \textbf{Rules for dev project network}: flow left to right, activity cannot begin until all predecessors complete, arrows on networks indicate precedence \& flow, each activity should have unique id number, id must be larger than predecessors, no loops \& conditionals, use common start \& stop nodes\\
        \textbf{Total slack}: amount of time activity can be delayed \& not delay project, time activity can exceed early finish date without affecting project end date/imposed completion date, LS --- ES or LF --- EF\\
        \textbf{Sensitivity}: likelihood original critical paths will change once project initiated\\
        \textbf{Free slack (float)}: amount of time activity can be delayed after start of longer parallel activities, how long activity can exceed EF without affecting ES of successors, allows flexibility in scheduling scarce resources\\
        \textbf{Laddering}: activities broken into segments so following activity can begin sooner \& not delay work\\
        \textbf{Lags}: minimum amount of time dependent activity must be delayed to begin/end, lengthy activities broken down to reduce delay in start of successor (if successor only dep on part finished early), with lags start and finish can have different slacks\\
        \textbf{Lag types}:\\
        \textit{Finish-to-start}: order materials, 1 day to place order \& 19 days to receive goods, can use for laddering\\
        \textit{Start-to-start}: can also be used for laddering, reduce network detail \& project delays, often used in concurrent engineering (instead of completely sequential, can start next stage once part of predecessor complete)\\
        \textit{Finish-to-finish}: test cannot be completed any earlier than 4 days after prototype complete, cannot be f-s because subcomponent test does not qualify as complete sys test, which takes 4 days\\
        \textit{Start-to-finish}: system doc cannot end until 3 days after test start, since relevant info is generated after 3 days of testing\\
        \textbf{Hammock activity}: spans over segment of project, used to aggregate sections of project to facilitate getting right amount of detail for specific sections of project, used to identify fixed resources/costs over segment of project (inspection services, consultants, construction management)\\
        \underline{\textbf{Risk}}: uncertain/chance events that planning cannot overcome/control, focus on future, deals with probabilities, tends to emphasise negative consequences\\
        \textbf{Threat}: risk event external to org (inflation, market acceptance, laws), not within project manager/team's responsibility area, normally considered before decision to proceed with project, if project initiated contingency funds placed in management reserve budget\\
        \textbf{Risk project life cycle}: high chance of risk occurring initially (defining \& planning) but low cost to fix, both swap by delivering\\
        \textbf{Risk anatomy}: cause (how \& why), event (what can go wrong, occurrence, outcome), effects (consequence)\\
        \textbf{Certainty}: knowns, decision-maker aware of alternatives \& outcomes\\
        \textbf{Uncertainty}: unknown unknowns, future unknowable to probabilities \& consequences unknown\\
        \textbf{Risk}: known unknowns, situation where future can be analysed \& planned for\\
        \textbf{Risk management attributes}: is a decision-making process (informs decisions), should have structure \& formality (helps effective management), has to have continuity through the project (iterative, continuous monitoring), has a project focus (for project performance \& outcomes, such as time, cost, perf)\\
        \textbf{Risk management benefits}: proactive rather than reactive approach, reduce surprises \& negative consequences, prepares project manager to take advantage of appropriate risks, provides better control over future, improves chances of reaching project perf objectives within budget \& on time\\
        \textbf{Risk management process}:\\
        \textit{Planning \& context}: defines factors (internal/external) to take into account, risk management plan\\
        \textit{Risk identification}: identify potential risks \& causes (list of risks)\\
        \textit{Risk analysis \& evaluation}: analyse risk likelihood \& potential consequences, risk evaluation for management\\
        \textit{Risk treatment}: strategies\\
        \textit{Implementation \& control}: implement, monitor, control, review, happens throughout\\
        \textbf{Risk management plan}: objectives, methodology, roles \& responsibilities, budgeting, timing, risk categories, scoring interpretation, tolerance thresholds, reporting formats, tracking\\
        \textbf{Context types}:\\
        \textit{External}: environment, such as political, social, legal, financial, geographical\\
        \textit{Organisational}: culture, values, governance, capabilities, policies, processes, strategic objectives\\
        \textit{Project}: fill set of objectives \& project outcomes\\
        \textbf{Risk Identification Tools}: personal xp, individual pondering, group processes, structured interviews, project info, checklists, risk breakdown structure\\
        \textbf{Risk Identification Method}: generate list of possible risks through brainstroming, problem identification \& risk profiling, focus on macro risks (affect whole project) then specific events, use core risk team and/or stakeholders, typically occurs in project planning phase, focus on actual events that could produce consequences rather than objectives (instead of `fail to meet deadline' focus on possible causes)\\
        \textbf{Risk breakdown structure}: split into categories (technical, external, organisational, PM), then split to subcategories\\
        \textbf{Risk evaluation}: need to do anything?, classify (how acceptable), determine risk tolerance, know risk appetite\\
        \textbf{Risk consequence matrix}: effect \& probability, contents are how important it is (insignificant almost certain is moderate)\\
        \textbf{Risk assessment form/matrix}: columns are risk events, likelihood, impact, detection difficulty, when\\
        \textbf{Risk severity matrix}: impact \& likelihood, with zones (red zone for high likelihood \& impact)\\
        \textbf{Failure Mode \& Effects Analysis}: add detection to severity matrix, risk value does not differentiate between what part contributes most\\
        \textbf{Risk treatment options}:\\
        \textit{Avoidance}: change plan to eliminate threat, refuse to accept risk\\
        \textit{Reduction/Mitigation}: reduce likelihood/consequences pre/post risk, contingency plans\\
        \textit{Retention}: accept with no further action, often for low risk\\
        \textit{Transfer}: shift responsibility \& consequences to another party (contract/insurance) though it still exists\\
        \textbf{Contingency plan}: alt plan that will be used if possible foreseen risk occurs, plan of actions to reduce/mitigate consequences of risk event, having no plan may slow managerial response, decisions made under pressure can be potentially dangerous \& costly\\
        \textbf{Technical Risks}: backup strategies if chosen tech fails, assessing whether tech uncertainties can be resolved\\
        \textbf{Schedule risks}: use of slack increase risk of late project finish, imposed duration dates (absolute project finish date), compression of project schedules due to shortened project duration date (crash or shortening project duration using contingency funds, run activities concurrently/laddered)\\
        \textbf{Costs risks}: costs increase then problem take longer to solve than expected (time/cost dependency links), price protection risks increase for long projects, evaluate item by item for cost sensitive projects\\
        \textbf{Funding risks}: changes in supply of funds can dramatically affect likelihood of implementation/successful completion of project\\
        \textbf{Risk response matrix}: columns are risk event, response, contingency plan, trigger, who is responsible\\
        \textbf{Principles for selection for treatment}:\\
        \textit{Practicality}: realistic, achievable, easy to implement\\
        \textit{Effectiveness}: rating comparative effectiveness of options\\
        \textit{Acceptability}: agreement \& commitment of stakeholders\\
        \textit{Cost}: balancing cost of treatment option against benefit\\
        \textit{Capability}: effective allocation for responsibility\\
        \textit{Timeliness}: implemented at the time to be successful\\
        \textit{Precautions}: need to take action as risk event has serious consequences\\
        \textbf{Time buffers}: amounts of time used to compensate for unplanned delays in project schedule, allocate at critical project times (activities with severe risk, merge activities that may become late due to predecessors being late, noncritical activities to stop them from becoming critical, activities that require scarce resources to ensure adequate time to get the resources), if overall schedule is uncertain could add at end of project but requires top management \& project owner authorisation\\
        \textbf{Implementation \& control}: use rpogress meetings \& risk audits to evaluate identified risks remain valid, any changes in level of risk, implementation process, new treatments identified, new risks identified\\
        \textbf{Risk register}: columns are number, risk, probability, consequences, rating, treatment, residual probability, residual consequences, residual rating, who, when, cost, status, contingency reseves through float (schedule) \& reserves (financial) to cover/reduce risk, updated at status meeting\\
        \textbf{Contingency funds}: funds to cover project risks --- identified \& unknown, size of funds reflects overall risk of project, use needs to be closely monitored, independent of original time/cost estimates (risk may not occur so not included in baseline, if occurs then draw and add to baseline, if not take out from reserve)\\
        \textbf{Contingency fund types}:\\
        \textit{Budget reserves}: linked to identified risks of specific work packages, allocated to specific work packages/activities, communicated to project team but allocated by PM if required\\
        \textit{Management reserves}: large funds used to cover major unforeseen risks (ex: change in scope), allocated to entire project, established after budget reserves identified \& funded, controlled by PM \& project owner (internal/external), may contain technical reserves for project involving highly innovative process/product as fallback plan in case process/product is unsuccessful\\
        \textbf{Opportunity management tactics}:\\
        \textit{Exploit}: seek to eliminate uncertainty associated with opportunity to ensure it definitely happens\\
        \textit{Share}: allocating some/all of ownership of opportunity who is best able to capture it for benefit of project\\
        \textit{Enhance}: take action to increase probability and/or positive impact of opportunity\\
        \textit{Accept}: be willing to take advantage if it occurs but not taking action to pursue\\
    \end{multicols}
    \end{document}
