\documentclass[a4paper]{article}
    \usepackage[margin=0in]{geometry}
    \usepackage{multicol}

    \begin{document}
    \begin{multicols}{2}
        \tiny
        \noindent\underline{\textbf{Project}}: complex, non-routine, one-time effort limited by time, budget, resources, perf specs designed to meet customer needs\\
        \textbf{Project characteristics}: established objective, defined life span with beginning \& end, requires participation across the org, typically involves doing sth never done before, has specific time, cost, perf reqs\\
        \textbf{Project life cycle}:\\
        \textit{Defining}: goals, specs, tasks, responsibilities\\
        \textit{Planning}: schedules, budgets, resources, risks, staffing\\
        \textit{Executing}: status reports, changes, quality, forecasts\\
        \textit{Closing}: train customer, transfer docs, release resources, evaluation, lessons learned\\
        \textbf{Factors leading to increased use of PM}: compression of product life cycle, knowledge explosion, triple bottom line (planet, people, profit), corporate downsizing, increased customer focus, small projects represent big problems\\
        \textbf{Project Manager}: manages temporary, non-repetitive activities, frequently acts independently of formal org, marshals resources for project, linked directly to customer interface, provides direction, coordination \& integration to the project team, responsible for performance \& success of the project, must induce right people at right time to address the right issues and make right decisions\\
        \textbf{PM Technical Aspects}: scope, WBS, schedules, resource allocation, baseline budgets, status reports\\
        \textbf{PM Sociocultural Aspects}: leadership, problem solving, teamwork, negotiation, politics, customer expectations\\
        \textbf{Integrated management of projects}: strategic alignment, portfolio management, PM, with org culture env wrapped around\\
        \textbf{Integration of projects with org strategy}: use of selection criteria to ensure strategic alignment and project priorities (effective use of org resources); selection process that is systematic, open, consistent \& balanced; all selected projects become part of portfolio that balances risk for org; portfolio management ensures only most valuable projects approved \& managed across entire org; value of project not only ROI but also strategic fit \& best use of org resources\\
        \textbf{Integrative PM approach benefits}: provide senior management with overview of all PM activities, big picure of how org resources used, risk assessment of project portfolio, rough metric of org's improvement in managing projects relative to others in industry, linkages of senior management with actual project execution management\\
        \textbf{Portfolio Management Functions}: oversee project selection, monitor aggregate resource levels \& skills, encourage use of best practices, balance projects in portfolio in order to represent risk level appropriate to the organisation, improve communication among all stakeholders, create total ord perspective that goes beyond silo thinking, improve overall management of projects over time\\
        \textbf{Program}: a series of coordinated, related, multiple projects that continue over an extended time and are intended to achieve a goal\\
        \underline{\textbf{Traditional PM}}: focus on thorough, upfront planning of entire project, requires high degree of predictability to be effective\\
        \textbf{Agile}: relies on incremental, iterative dev cycles to complete less predictable projects, ideal for exploratory projects in which requirements need to be discovered and new tech tested (uncertain ablut how long, what is required, allows change in reqs), focus on active collaboration between project tem \& customer reps\\
        \begin{tabular}{l l}
            \textbf{Traditional} & \textbf{Agile}\\
            Design up front & Continuous design\\
            Fixed scope & Flexible\\
            Deliverables & Features/reqs\\
            Freeze design as early as possible & as late as possible\\
            Low uncertainty & high\\
            Avoid change & embrace\\
            Low customer interaction & high\\
            Conventional project teams & self-organised\\
        \end{tabular}\\
        \textbf{Agile Details}: use iterations to develop workable product that satisfies the customer and other key stakeholders, stakeholders \& customers review progress \& re-evaluate priorities to ensure alignment with customer needs \& company goals, adjustments are made \& a different iterative cycle begins that subsumes the work of the previous iterations \& adds new capabilities to the evolving product\\
        \textbf{Agile Advantages}: useful in developing critical breakthrough tech or defining essential features; continuous integration, verification \& validation of the evolving product; frequent demonstration of progress to increase likelihood that end product will satisfy customer needs; early detection of defects \& problems\\
        \textbf{Agile Limitations}: does not satisfy top management's need for budget, scope \& schedule control; self-organisation \& close collaboration principles can be incompatible with corporate cultures; appears to work best on small project with 5--9 people, requires active customer involvement \& cooperation\\
        \textbf{Agile Principles}: focus on customer value, iterative \& incremental delivery, experimentation \& adaptation, self-organisation, continuous improvement\\
        \textbf{Project uncertainty dimensions}: scope \& tech
        \textbf{Scrum}: holistic (interconnected emphasis) approach for use by cross-functional team collaborating to develop new product, defines product features as deliverables \& prioritises them by perceived highest value to the customer, re-evaluates priorities after each iteration/sprint to produce fulle functional features, phases: analysis, design, build, test\\
        \textbf{Scrum roles \& responsibilities}:\\
        \textit{Product owner}: acts on behalf of customer to represent interests, responsible for product backlog priorities \& process selection\\
        \textit{Development team}: 5--9 people with cross-functional skillsets responsible for delivering product, sets own goals, organises itself, makes decisions\\
        \textit{Scrum master}: facilitates scrum process and resolves impediments at the team \& org level by acting as buffer between team \& outside interference\\
        \textbf{Scrum practices}:\\
        \textit{Sprint}: time-controlled mini-project that implements specific portion of a system, 30 day time box with specific goals \& deliverables, frozen scope defined from sprint backlog\\
        \textit{Daily Scrum}: daily meeting of all team members to report progress (15 min max), also called standup\\
        \textit{Sprint final half-day review meeting}: review \& identify changes needed for following sprints\\
        \textbf{Sprint meetings}: sprint planning, daily scrum, sprint review, sprint retrospective\\
        \textbf{Product backlog}: customer's prioritised list of desired key features for the completed project, can only be changed by product owner\\
        \textbf{Sprint backlog}: amount of work team commits to complete during the next sprint, developed \& controlled by team\\
        \textbf{Scaling}: using several teams to work on different features of large scale project at same time\\
        \textbf{Staging}: upfront planning to manage interdependencies of the different features to develop, involves developing protocols \& defining roles to coordinate efforts \& assure compatibility \& harmony\\
        \underline{\textbf{Strategy Importance}}: Project managers must respond to changes to organisation mission and strategy appropriately, if understand strategy can become effective advocates of projects aligned with firm's mission\\
        \textbf{Mistakes by not understanding role of projects in accomplishing strategy}: focus on problems/solutions with low strategic priority, focus on immediate customer rather than whole marketplace \& value chain, over-emphasising tech that results in projects that pursue exotic tech that does not fit strategy or customer need, trying to solve customer issues with product/service rather than focusing on 20\% with 80\% of value (\textbf{Pareto's law}), engaging in never-eding search of perfection that only team really cares about\\
        \textbf{Strategic management}: requires every project to be clearly linked to strategy; provides theme \& focus of organisational future direction (responding to changes in external env --- env scanning, allocating scarce resources of firm to improve competitive position --- internal responses to new programs); requires strong links among mission, goals, objectives, strategy, impl\\
        \textbf{Strategic management activities}:\\
        \textit{Review \& define org mission}: identify \& communicate purpose of org to stakeholders, identify scope of org in terms of product/service, provides focus for decision making, used for eval org perf\\
        \textit{Set long-range goals \& objectives}: translate mission to specific, concrete \& measurable terms; sets targets for all levels of org in a cascaded manner; where is org headed and when it will get there; focus managers on where org should move to\\
        \textit{Analyse \& formulate strategies to reach objectives}: focus on what needs to be done to reach objectives, relaistic view of past \& current position, SWOT analysis, alternatives generated \& assessed, strategy formulation \& assignation\\
        \textit{Implement strategies through projects}: focus on how strategies will be realised with resources, maintain link between strategy (what) \& impl (how), requires resource allocation, action \& completion of tasks, prioritisation\\
        \textbf{SMART objectives}: Specific, Measurable (indicators of progress), Assignable (to one person for completion), Realistic (what can realistically be done with avail resources), Time related (state when objective can be achieved)\\
        \textbf{SWOT analysis}: internal (strengths, weaknesses) \& external (opportunities, threats) analysis\\
        \textbf{Scenario planning}: longer term, steps: clarifying core business \& assessing drivers of change in industry env, dev potential scenatios \& assess impact of STEEP factors, dev potential contingency strategies \& best future strategic options, identifying early indicators \& establishing triggers for strategic action\\
        \textbf{STEEP factors}: social, tech, env, economic, political\\
        \textbf{Project portfolio management benefits}: build discipline to project selection process, link project selection to strategic metrics, prioritise project proposal across common set of criteria rather than politics/emotion, allocate resources to projects that align with strategic direction, balance risk across all projects, justifies stopping projects that don't support strategy, improves comms \& supports agreement on project goals\\
        \textbf{Project portfolio management problems}:\\
        \textit{Implementation gap}: lack of understanding \& consensus on strategy among to management \& middle-level (functional) managers who independently implement strategy\\
        \textit{Org politics}: project selection based on persuasiveness \& power of people advocating projects\\
        \textit{Resource conflicts \& multitasking}: Multiproject env creates interdependency relationships of shared resources which results in starting, stopping \& restarting of projects\\
        \textbf{Project portfolio sys design}: classification of project, selection criteria depending upon classification, sources of proposals, evaluating proposals, ranking proposals, managing portfolio of projects\\
        \textbf{Project types}: compliance (must-do, incl emergency, meet regulations, usually have penalties if not impl), strategic (directly support long-run mission, increase revenue/market share, ex: new products, R\&D), operational (support current ops, improve perf, reduce product cost, improve efficiency of delivery sys, ex: upgrade building green rating)\\
        \textbf{Financial Selection Criteria}: payback, NPV, IRR (internal rate of return, inverse of payback)\\
        \textbf{Payback model}: measures time project takes to recover investment; uses more desirable, shorter paybacks; emphasises cash flows (key factor in business)\\
        \textbf{Payback limitations}: ignores time value of money, assumes casf inflow only for investment period, does not consider profitability\\
        \textbf{Net Present Value}: $I_0 + \sum\nolimits_{t=1}^n\frac{F_t}{{(1 + k)}^t}$, $I_0$ is initial investment (negative), $F_t$ is net cash inflow for period $t$, $k$ is required rate of return, want positive\\
        \textbf{Non-financial strategic criteria}: capture larger market share, make it difficult for competitors to enter the market, develop enabler product which by interduction will increase sales in more profitable products, develop core tech to be used in next-gen products, reduce dependency on unreliable suppliers, prevent government intervention \& regulation\\
        \textbf{Multicriteria selection models}:\\
        \textit{Checklist model}: use list of questions to review potential projects \& to determine accept/reject, fails to answer relative importance/value of potential project \& doesn't allow for comparison with others\\
        \textit{Multiweighted scoring model}: use several weighted qualitative and/or quantitative selection criteria to evaluate project proposals, can use for comparison\\
        \textbf{Selection model advantages}: bring projects to closer alignment with org strategic goals, reduce number of wasteful projects, help identify proper goals for projects, help everyone involved understand how \& why project is selected\\
        \textbf{Project relativity matrix}: 2 dimensions (technical feasibility, NPV), white elephant (low, low, showed promise at one time but are no longer viable), oyster (low, high, technological breakthroughs with high commercial payoffs), bread-and-butter (high, low, evolutionary improvements to current products \& services), pearl (high, high, revolutionary commercial opportunities using proven tech advances)\\
    \end{multicols}
    \end{document}
