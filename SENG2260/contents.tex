\documentclass[a4paper]{article}
    \usepackage[margin=0in]{geometry}
    \usepackage{multicol}

    \begin{document}
    \tiny
    \begin{multicols}{3}
        \noindent\textbf{HCI According to ACM}: the discipline concerned with design, eval, impl of interactive computer sys for human use \& with study of major phenom surrounding them\\
        \textbf{HCI Definition}: study of interaction between people \& computer based sys, concerned with phys, physiological \& theoretical aspects of this proc, about designing computer sys that support people so that they can carry out their acts productively \& safely\\
        \textbf{User}: individual user, group of users working together or a seq of users in org dealing with some part of proc/task\\
        \textbf{Computer}: tech ranging from desktop to large scale sys, or control/embedded sys\\
        \textbf{Interaction}: comms between user \& computer in direct/indirect manner\\
        \textbf{What is involved}: study of humans using intf, dev of new apps/sys to support user's acts, new devices \& tools for users, develop usable prod (easy to learn, effective to use, provide enjoyable \& satisfying xp)\\
        \textbf{Interdisciplinary (HCI)}: compsci \& sys design are central concerns, but not possible to design effective interactive sys from one discipline in isolation\\
        \textbf{Contributing Disciplines}: cognitive psych, compsci, anthropology, engg, ergonomics \& human factors, design, social \& organizational psych, sociology, philosophy, AI, linguistics\\
        \textbf{Importance}: how to make sys usable, evaluate usability of bespoke (custom) sys, understanding how users interact with computers \& enabling users to do so effectively, matter of law (is suitable to task, easy to use \& adaptable to user's knowledge \& xp, provides feedback on perf, displays info in format \& pace adapted to user, conforms to principles of software ergonomics)\\
        \textbf{Factors in HCI}: organisational, environmental, health \& safety, the user, comfort, UI, task factors, constraints, sys func, productivity factors, more:\\
        \textbf{Human}: human info processing, lang, comms \& interaction, ergonomics\\
        \textbf{Computer}: I/O devices, dialogue techniques, dialogue genre, computer graphics, dialogue arch\\
        \textbf{Dev Process}: design approaches, impl techniques \& tools, eval techniques, example sys \& case studies\\
        \textbf{Problems with Software}: excessive \& unwanted share dealing in stock market, error in dosage given to patients receiving rad therapy, erratic behaviour of military \& civil aircraft, difficulties in controlling nuclear power plants during sys failures, delays in dispatching ambulances to accidents\\
        \textbf{Avoiding Problematic Design + what to design}: take into account who users are, what acts are being carried out, where interaction is taking place, optimise interactions to they match users' acts \& needs\\
        \textbf{3U}: Useful (accomplish what is required), Usable (do it easily \& naturally, without danger of error), Used (make people want to use it, be attractive, engaging, fun)\\
        \textbf{Principles for supporting HCI + understanding users' needs}: take into account what people are good \& bad at, consider what might help people with the way they currently do things, think through what might provide quality user xp, listen to what people want \& getting them involved in design, using tried \& tested user based techniques during design proc\\
        \textbf{Science or Craft}: bit of both (artistically pleasing \& capable of fulfilling tasks required), innovative ideas lead to more usable sys (understand not only that they work but how \& why they work), creative flow underpinned with science, scientific method accelerated by artistic insight\\
        \underline{\textbf{Bad Design}}: button \& label look the same, buttons on different sides laid out different, need to push button first to activate instead of inserting bill (against convention)\\
        \textbf{Good Design}: marble answering machine (based on how everyday objs behave, easy, intuitive \& pleasure to use, one step acts to perform core tasks), TiVo remote (peanut shape to fit in hand, logical layout, color coded distinctive buttons, easy to locate buttons)\\
        \textbf{Interaction Design (ID)}: designing interactive prod to support the way people comm \& interact in their everyday \& working lives (Preece, Sharp, Rogers, 2015), design of spaces for human comms \& interaction (Winograd, 1997)\\
        \textbf{ID Goals}: develop usable prod (easy to learn, effective to use \& provide an enjoyable xp), involve users in design proc\\
        \textbf{Interdisciplinary Contributor (ID)}: academic (psych, social sci, computing, engg, ergonomics, informatics), design (graphic, prod, artist, industrial, film industry)\\
        \textbf{Interdisciplinary Fields Doing ID}: HCI, ubiquitous computing, human factors, cognitive engg, cognitive ergonomics, computer supported cooperative work, info sys\\
        \textbf{Working in Multidisciplinary Teams}: many people from different backgrounds involved, different perspectives \& ways of seeing \& talking about things, more ideas \& designs generated, but difficult to comm \& progress forward the designs being created\\
        \textbf{ID in Business}: help companies enter age of consumer, design human centered prod \& services, from research \& prod to goal related design, provides wide range of design services, in each case targeted to address prod dev needs at hand, creates prod, services, environmments for companies pioneering new ways to provide value to customers\\
        \textbf{Professionals in ID}: interaction designers (design of interactive aspects), usability engineers (evaluate prod using usability methods \& principles), web designers (develop \& create visual design of websites, such as layouts), info architects (people who come up with ideas of how to plan \& struct interactive prod), UX designers (all before + field studies to inform prod design)\\
        \textbf{UX}: how prod behaves \& is used by people in real world, way people feel about it \& their pleasure \& satisfaction when using it, looking at it, holding it, opening/closing it, cannot design UX, only design for UX\\
        \textbf{iPod}: quality UX from start, simple, elegant, distinct brand, pleasurable, must have fashion item, catchy names, cool\\
        \textbf{ID Process}: establishing reqs, dev alts, prototype, evaluate\\
        \textbf{ID Characteristics}: users should be involved through dev of the project, specific usability \& user xp goals need to be identified, clearly documented \& agreed at the beginning of the project, iteration needed through core acts\\
        \textbf{Help Designers}: understand how to design interactive prod that fit with what people want, need desire, appreciate that one size does not fir all, identify incorrect assumptions about particular user groups (not all old people want/need big fonts), be aware of both people's sensitivities \& capabilities\\
        \textbf{Cultural Differences}: date format\\
        \textbf{Accessibility}: degree to which prod usable \& accessible by as many people as possible, focus on disability (mental or phys impairment, adverse effect on everyday life, long term)\\
        \textbf{Usability Goals}: effective, efficient, safe to use, have good utility, easy to learn, easy to remember how to use\\
        \textbf{Design Principles}: visibility (of act, invisible auto controls can be more difficult to use), feedback (sending info back to user), constraints (help prevent user from selecting incorrect options), consistency (easier to learn \& use, but can break down, ex: control + first letter but problem if multiple commands with same first letter, internal is in app, external is across apps \& devices), affordances (attribute of obj that allows people to know how to use it, virt intf better conceptualized as perceived affordances, learned conventions of arbitrary mappings between act \& effect)\\
        \underline{\textbf{Understanding Problem Space}}: what you want to create, assumptions, will it achieve what you hope it will\\
        \textbf{Assumption}: taking something for granted when it needs further investigation (bad: people want to watch TV while driving, wouldn't mind payng a lot more for 3D TV, ok: wouldn't mind wearing 3D glasses in living rooms, enjoy enhanced clarity \& color detail from 3D)\\
        \textbf{Claim}: state something to be true when it is still open to question (voice commands for GPS is safe)\\
        \textbf{Framework for analysing problem space}: are there probs with existing prod/UX, why there are probs, how do you think proposed design ideas might overcome these, if designing for new UX how proposed design ideas support, change, extend current ways of doing things\\
        \textbf{Benefits of Conceptualising}: orientation (enable design teams to ask specific qs about how conceptual model will be understood), open minded (prevent design teams from becoming narrowly focused early), common ground (allow design teams to establish set of commonly agreed terms)\\
        \textbf{From Problem to Design Space}: understand problem can help inform design (what kind of intf, behaviour, func to provide), important to develop CM before\\
        \textbf{Conceptual Model (CM)}: high level description of how sys organized \& operates, enables designers to straighten out thinking before they start laying out their widgets (Johnson, Henderson, 2002), describe in terms of core acts \& objs, also in terms of intf metaphors\\
        \textbf{CM components}: metaphors \& analogies (understand what prod is for \& how to use it for act), concepts people are exposed to through prod (task-domain objs, attributes, ops: saving, revisiting, organizing), rel \& mappings between concepts\\
        \textbf{First steps in formulating CM}: what will users be doing when carrying out tasks, how will sys support tasks, what kind of intf metaphor (if any) will be appropriate, what kinds of interaction modes \& styles to use, always keep in mind when making design decisions how user will understand underlying conceptual model\\
        \textbf{Interface Metaphors}: conceptualizing what we are doing (surf web), conceptual model instantiated at the intf (desktop metaphor), visualizing op (icon of shopping cart for placing items into), designed to be similar to phys entity but also has own properties, can be based on act, obj or both, exploit user's familiar knowledge to help then understand unfamiliar, conjures up essence of unfamiliar act, enabling users to leverage thos to understand more aspects of unfamiliar func ex: material (card, familiar form factor, material properties added, giving appearance \& phys behaviour, like surface of paper)\\
        \textbf{Interface Metaphor Benefits}: makes learning new sys easier, helps users understand underlying conceptual model, can be very innovative \& enable realm of computers \& apps to be made more accessible to greater diversity of users\\
        \textbf{Interface Metaphor Problems}: break conventional \& cultural rules (recycle bin on desktop), can constrain designers in way they conceptualize problem space, conflict with design principles, forces users to only understand sys in terms of meethphor, designers can inadvertently use bad existing designs \& transfer bad parts over, limit designers' imagination with coming up with new conceptual models\\
        \textbf{Interaction Types (CM)}: hybrid often used, support different ways to do same thing, can take longer to learn, more below
        \textbf{Instructing}: issue command, select option, quick \& efficient interaction, good for repetitive kinds of acts performed on multiple objs, ex: word processor, vending machine\\
        \textbf{Conversing}: underlying model of conversation with human, range from simple voice recog menu driven sys to more complex natural lang dialogs, virt agents, toys, robots designed to converse, ex: timetables, search engines, advice giving sys, help sys, allows users (especially novice \& technophobes) to interact with sys in way that is familiar (make them feel comfortable, at ease, less scared), but misunderstanding can arise when sys cannot parse what user says\\
        \textbf{Manipulating}: involves drag, select, open, zoom on virt objs, exploit user's knowledge of how they mode \& manipulate in phys world, can involve acts using phys controllers (Wii) or air gestures (Kinect) to control movements of on screen avatar, tagged phys objs that are maniplulated in phys world result in phys/digi events (animation)\\
        \textbf{Direct Manipulation (DM)}: continuous representation of objs \& acts of interest, phys acts \& button pressing instead of issuing commands with complex syntax, rapid reversible acts with immediate feedback on obj of interest\\
        \textbf{DM Advantages}: novice can learn basic func quickly, experienced users can work extremely rapidly to carry out wide range of tasks (even defining new functions), intermittent users can retain operational concepts over time, error msgs rarely needed, users can immediately see if acts are furthering goals \& if not do something else, users xp less anxiety, users gain confidence \& mastery \& feel in control\\
        \textbf{DM Disadvantages}: some people take metaphor too literally, not all tasks can be described by objs \& not all acts can be done directly, some tasks better achieved through delegating (spell checking), can become screen space gobblers, moving mouse can be slower than pressing func keys to do same acts\\
        \textbf{Exploring}: involves users moving through virt or phys environments (with embedded sensor tech)\\
        \textbf{Interface Types (CM)}: kind of intf used to support mode, ex: command, speech, data entry, form fill in, query, graphical, web, pen, VR/AR/Mixed, gesture, brain, when choosing need to determine reqs \& user needs, take budget \& other constraints into account, also depend on suitability of tech for act being supported\\
        \textbf{Paradigm}: inspiration for CM, general approach adopted by community for carrying out research (shared assumptions, concepts, values, practices), ex: ubiquitous computing, pervasive computing, wearable computing, tangible bits, AR, attentive environments, ambient computing\\
        \textbf{Visions}: driving force that frames R\&D, invites people to imagine what life will be like in 10, 15, 20 years time (Apple 1987 Knowledge Navigator, Smart Cities, Smart health), provide concrete scenarios of how society can use next gen of imagined tech, also  raise qs concerning privacy \& trust\\
        \textbf{Theory}: explanation of phenom (info processing that explains how the mind or some aspect of it is assumed to work), can help identify factors (cognitive, social, affective, relevant to design \& eval of interactive prod)\\
        \textbf{Models}: simplification of HCI phenom, intended to make it easier for designers to predict \& evaluate alt designs, abstracted from theory from contributing discipline (psych, keystroke model)\\
        \textbf{Framework}: set of interrelated concepts \&/or specific qs for `what to look for', provide advice on how to design\\
        \underline{\textbf{Why Need to Understand Users}}: interacting with tech is cognitive (need to take into account cognitive processes involved \& cognitive limitations of users, provides knowledge about what users can \& cannot be expected to do, identifies \& explains nature \& causes of probs users encounter, supply theories, modelling tools, guidance \& methods that can lead to design of better interactive prod)\\
        \textbf{Cognitive Processes}: below\\
        \textbf{Attention}: select things to concentrate on at point in time from mass of stimuli around, allow to focus on info relevant to what we are doing but limits ability to keep track of all events, involves audio \&/or visual senses, info at intf should be structured to capture attention (use perceptual bounds like windows, colour, sound, ordering, spacing, underlining, sequencing, animation), multitaskers easily distracted \& find it hard to filter irrelevant info\\
        \textbf{Design Implications for Attention}: make info noticeable when it needs attending to, use techniques that make things stand out, avoid clutter intf with too much info (search engine \& form fill in that have simple \& clean intf easier to use)\\
        \textbf{Perception}: how info acquired from world \& transformed to xps, text should be legible, icons should be easy to distinguish \& read, group info\\
        \textbf{Design Implications for Perception}: icons should enable users to readily distinguish meaning, bordering \& spacing are effective visual ways of grouping info, sounds should be audible \& distinguishable, speech output should enable users to distinguish between set of spoken words, text should be legible \& distinguishable from background, tactile feedback should allow users to recognize \& distinguish different meanings\\
        \textbf{Memory}: involves first encoding (more attention paid, more it is processed in terms of thinking \& comparing with other knowledge, more likely to be remembered), then retrieving knowledge, don't remember everything (involves filtering \& processing what is attended to), context important in affecting memory (where, when), recognize better than recall, remember less about photographed objs than actually seen (Henkel, 2014), people can scan list to find one they want\\
        \textbf{Digital Content Management}: memory involves 2 processes (recall directed \& recog based scanning), file mngt sys should be designed to optimize both (search box \& history), help users encode files in richer ways (colour, flagging, image, flexible text, timestamp)\\
        \textbf{SenseCam}: intermittently takes photos without user intervention while worn, can improve memory from Alzheimer's\\
        \textbf{Design Implications}: don't overload memory with complicated procedures, recog vs recall, provide various ways of encoding info\\
        \textbf{Learning}: prefer to learn by doing rather than read manual, rely more on internet to look things up, expecting to have internet reduces need \& extent to which we remember, enhances memory of knowing where to find it online (Sparrow et al, 2011)\\
        \textbf{Design Implications}: design intf that encourage exploration, constrain \& guide learners, dynamically linking concepts \& representations can facilitate learning of complex material\\
        \textbf{Reading, Speaking, Listening}: many prefer listening to reading, reading can be quicker, listening requires less cognitive effort, dyslexics have difficulties understanding \& recognizing written words, speech recog, speech output, natural lang sys (type in qs \& give text based responses)\\
        \textbf {Design Implications}: speech based menus \& instructions should be short, accentuate intonation of artificial voice, provide opportunities for making text large on screen\\
        \textbf{Problem Solving, Planning, Reasoning, Decision Making}: involves reflective cognition (thinking about what to do, what the options are, consequences), often involves conscious processes, discussion with others or self, use of artefacts (maps, books, pen, paper), may involve working though different scenarios \& deciding which is best option\\
        \textbf{Design Implications}: provide additional info/functions for users who wish to understand more about how to carry out an act more effectively, use simple computational aids to support rapid decision making \& planning for users on the move\\
        \textbf{App Mentality}: developing in psyche of younger generation is making it worse for them to make their own decisions because they are becoming risk averse (Gardner, Davis, 2013), all desires \& qs should be satisfied/answered by app, so less thinking?\\
        \textbf{Mental Model}: how to use the sys (what to do next), what to do with unfamiliar sys or unexpected situations (how sys works), used to make inferences, involves unconscious \& conscious processes (images \& analogies activated), deep (how to drive car) vs shallow (how car works), errorneous ex: turn up thermostat to heat up quicker\\
        \textbf{Gulf of Execution \& Evaluation}: exec is from user to sys, eval is other way\\
        \textbf{Info Processing Steps}: encoding, comparison, response select, exec\\
        \textbf{Distributed Cognition (DC)}: info transformed through different media (computer, display, paper, head) instead of in mind only\\
        \textbf{DC Involves}: distributed problem solving, role of verbal \& on verbal behaviour, various coordinating mechanisms used (rules, procedures), comms that takes place as collaborative act progresses, how knowledge shared \& accessed\\
        \textbf{External Cognition}: concerned with explaining how we interact with external representations (maps, notes, diagrams), what are cognitive benefits, what processes involved, how they extend cognition, what computer based representations can we develop to help even more\\
        \textbf{Externalizing to reduce memory load}: remind that we need to do something, remind what to do, remind when to do, ex: diaries, reminders, calendars, notes, shopping \& todo lists, post-its, piles, marked emails\\
        \textbf{Computational Offloading}: when tool is used in conjunction with external representation to carry out computation (pen \& paper)\\
        \textbf{Annotation}: modify existing representations through making marks (cross off, tick, underline)\\
        \textbf{Cognitive Tracing}: externally manipulating items into different orders/structs (playing Scrabble, cards)\\
        \textbf{Design Implication}: provide external representations at intf that memory load \& facilitate computational offloading (info visualizations to allow people to make sense of \& make rapid decisions about big data)\\
        \underline{\textbf{What is Proto}}: screen sketches, storyboard, ppt, vid simulating use, lump of wood (for size), cardboard, limited function SW written in target lang or other\\
        \textbf{Why Proto}: eval \& feedback central to interaction design, stakeholders can see, hold, interact with proto more easily than doc/drawing, team members can comm effectiely, test ideas, encourages reflection, answer qs and support designers in choosing between alts\\
        \textbf{Proto Filtering Dimensions}: appearance, data, func, interactivity, spatial struct (layout)\\
        \textbf{Proto Manifestation Dimensions}: material, resolution/fidelity, scope\\
        \textbf{What to Proto}: technical issues, workflow, task design, screen layouts \& info display, difficult, controversial, critical areas\\
        \textbf{Low-F Proto}: use medium unlike final, quick, cheap, easily changed, ex: storyboards (sketch series, early), card based (often for web), wiz of oz (dev acting as sys, hidden)\\
        \textbf{High-F Proto}: use materials expected to be in final, look more like final, can be dev by integrating existing HW \& SW components, danger that users think they have complete system\\
        \textbf{Proto Compromises}: SW based proto may have slow response, sketchy icons, limited func, horizontal (wide range of func, little detail), vertical (lot of detail, few func)\\
        \textbf{Conceptual Design}: transform user req/needs to CM, consider alts (proto helps)\\
        \textbf{Metaphor Eval}: how much struct does it provide, how much is relevant to prob, is it easy to represent, will users understand, how extensible is it\\
        \textbf{Expanding Init CM}: what func will prod do (what prod do \& what user do), how func related to each other (seq or parallel, categories), what info needed (data req for task, how data transformed by sys)\\
        \textbf{Concrete Design}: many aspects (colour, icons, buttons, interaction devices), user characteristics \& context (accessibility, cross-cultural design), successful products are bundles of social solutions, inventors succeed in particular culture because they understand values, institutional arrangements, economic notions of culture\\
        \textbf{Using Scenarios}: express proposed or imagined situations, used in various ways (basis for overall design, script for user eval of proto, concrete ex of tasks, as means of coop across prof boundaries), to explore extreme cases\\
        \textbf{Explore UX}: use card based proto or stickies to model UX, called design/customer/XP/user journey map, can be wheel or timeline\\
        \textbf{Proto Construction}: phys (electronics, Arduino), SDK\\
        \underline{\textbf{Emotions and UX}}: HCI traditionally been about designing efficient \& effective sys, now more about how to design sys that make ppl respond in certain ways (happy, trusing, learn, motivated)\\
        \textbf{Emotional Interaction}: what makes us feel stuff, why ppl become emotionally attacked to certain prod (virt pet), can social bot help reduce loneliness and improve wellbeing, how to chg human behaviour through use of emotive feedback, emotional state changes how we think (if frightened/angry more likely to be less tolerant, if happy more likely to overlook minor prob \& be more creative)\\
        \textbf{Expressive Intf}: provide reassuring feedback that can be both informative \& fun (but can also be intrusive, annoying or making ppl angry), colour, sounds, icons, graphic elements \& anim used to make look \& feel of intf appealing (convey emotional state), can affect usability (ppl prepared to put up with certain aspects of intf if end result is appealing \& aesthetic, ex: large graphics slow dl)\\
        \textbf{Friendly Intf}: 3D metaphors based on familiar places (living rooms), agents (bunny, dog) incl to talk to user, make user feel more at ease \& comfortable\\
        \textbf{Frustrating Intf}: app doesn't work properly/crash, sys doesn't do what user wants it to do, unmet expectations, sys does not provide enough info to enable user to know what to do, vague/obtuse/condemning error, appearance of intf garish, noisy, gimmicky, patronizing, sys req users to do too many steps for task, then discover mistake made earlier and need to start all over\\
        \textbf{Gimmick}: amusing to designer but not to user (site under construction)\\
        \textbf{Error Msg}: avoid fatal, invalid, bad, audio warnings, uppercase and code numbers, vague, provide context-sensitive help\\
        \textbf{Emotional Tech}: measure facial expr, gesture, body movement, aim to predict user's emotions and aspects of behaviour (what user most likely to buy online when feeling sad, bored, happy)\\
        \textbf{Facial Coding}: measures user's emotions as they interact with computer/tablet, analyse img captured by cam of face, use to gauge engagement when looking at movies, online shopping, ads\\
        \textbf{Emotional Data Use}: adapt content to match user's emotional state\\
        \textbf{Indirect Emotion Detection}: beginning to be used more to infer/predict behaviour\\
        \textbf{Persuasive Tech}: designed to change attitude \& behaviour (Fogg, 2003), referred to as nudging (pop up ads, warning msg, reminder, prompt, personalized msg, recom, Amazon 1-click, pocket Pikachu, tracking devices, visualize electricity usage)\\
        \textbf{Anthromorphism}: attributing human like qualities to inanimate objects, much exploited in HCI (make UX more enjoyable, more motivating, make ppl feel at ease, reduce anxiety)\\
        \textbf{Anthromorphism Pros, Cons}: positive impact, more willing to continue with kind feedback, but deceptive, make ppl feel anxious, inferior, stupid, many prefer impersonal, make users feel less responsible for actions\\
        \textbf{Virt Chars}: sales agents, game char, learning companions, wizards, pets, newsreaders, provides welcoming persona, has personality and makes user feel involved, but can lead ppl into false sense of belief (confide personal secret), annoying and frustrating, may not be trustworthy\\
        \textbf{Virt Agent Believable}: extent to which users believe agent's intentions and personality, appearance and behaviour\\
        \underline{\textbf{Global Intf Issues}}: individual's response affected by factors (age, gender, race, sexuality, class, religion, political persuasion), exported SW needs modifications to suit local customs, laws, conventions, dev of multiple intf costly, so need to make generic \& easily modifiable intf\\
        \textbf{Intf Specialization Lvl}: globalisation (cultureless std), internationalisation (design base struct with intent of layer customisation), localisation (dev specific intf for paritcular market, translation, gov regulations, business practices, brand mngt, cultural elements, seamless integration)\\
        \textbf{Effective Design}: recognize cultural elements in given app, cultural diversity makes it unrealistic for designers to rely on intuition or personal xp for intf design, adaptation of shared intf requires identification of user factors, incl objective (gender, age, ethnic background, mother-tongue), subjective (cognitive style)\\
        \textbf{Approaches}: adoption of user centred dev approach (users and devs work together on identification of factors affecting usability and perf), effective use of iterative \& parallel prorotyping (facilitate user participation and maximise effectiveness of intf eval process while minimise time req for dev), integration of Taguchi techniques (to provide rigour for identification of optimum intf, minimise variation as main means for inproving quality), systematic and logical integration of techniques (so method can be applied by commercial intf designers)\\
        \textbf{Cultural Factors}: power distance (degree of dependence between boss and subordinate), collectivism-individualism (integration into cohesive groups or being expected to look after self), femininity-masculinity (extent gender roles distinct or overlap), uncertainty avoidance (extent feel threatened)\\
        \textbf{Cultural Model}: help identify levels of issues being involved in cmplx prob by using international var (categories that org cultural data) or dimensions of culture, meta models help understand how and where culture comes to influence lives in profound way\\
        \textbf{Iceberg Model}: only 10 percent of cultural characteristics of target audience is easily visible to observer, surface (number, currency, time, date format), unspoken rules (obscured, need context of situation), unconscious rules (difficult to study)\\
        \textbf{Multicultural Intf Design}: charset (alphabet), collating seq (character sorting order for list), format (number, date, time, currency), layout (address, tel no), icons (red cross not recognisable as hospital/health), symbols (x means not wanted, but sometimes used to fill box), colours, screen text, menu accelerators (on translation maybe two commands start with same letter)\\
        \textbf{Cultural Design Considerations}: identify lang \& country, address gov regulations, support charsets, international db design, code form so data not corrupted, display info in culturally correcy ways, use appropriate currency, design graphical img with care, provide alts for natural lang proc and audio/video elements\\
        \textbf{Single App Localisation}: handle multi lang within same app, consistent but localised UI\\
        \textbf{Avoid}: hard code text, refer to culture specific std, use slang, jargon, humor, sarcasm, colloquialism, metaphors, form plurals by adding (s) (use both forms if needed)\\
        \textbf{User Customisation}: keep sentences short and simple, allow users to select date and time format, calendar format, numeric and monetary format, paper size\\
        \underline{\textbf{Usability Testing}}: goals \& qs focus on how well users perform tasks with prod, comparison of prod or proto common, focus on time to complete task \& type of errors, testing central, recording perf of typical users doing typical tasks, controlled settings, users observed and timed, data record on vid, log key press, data used to calc perf time identify \& explain errors, user satisfaction eval using qs and interviews, field observations may be used to provide contextual understanding\\
        \textbf{UT vs Research}: UT:\@ improve prod, few participants, results inform design, usually not completely replicable, cond controlled as much as possible, procedure planned, results reported to devs; Research: discover knowledge, many participants, results validated statistically, must be replicable, strongly controlled cond, experimental design, sci report to sci community\\
        \textbf{Testing Conditions}: usability lab or other controlled space, emphasis on selecting representative users and developing representative tasks, tasks usually around 30 min, same test conditions for all, informed consent form explains procedures and deals with ethical issues\\
        \textbf{Types of Data}: time to complete task (also after time away from prod), n \& type of errors per task, n of errors per time, n of times online help/manuals accessed, n of users making error, n of users successfully completing task\\
        \textbf{N of Participants}: dep on sched for test, availability of participants, cost of running tests, typically 5 to 10, some argue to continue until no new insights\\
        \textbf{iPad Usability Test}: 7 particpants with 3+ months xp with iPhone, sign consent form (what to do, length of time, compensation, right to withdraw at any time, promise that identity would not be disclosed, agreement that data collected confidential and available to evaluators only), then explore iPad, then perform random assigned tasks (download \& read ebook, find stuff to buy, browse mag and find best pic of week, find recipe, find hotel)\\
        \textbf{Experimental Designs (Participants)}: diff (single group alloc randomly to experimental cond, no order eff but indiv diff problem), same (all appear in both cond, no indiv diff, but need to counter balance order eff), matched (in pairs, based on expertise, gender, no order effect and indiv diff reduced, but cannot be sure of perfect match)\\
        \textbf{Field Studies}: natural settings, aim to understand what users do naturally and how tech impacts them\\
        \textbf{Field Study Uses}: identify opportunities for new tech, determine design req, decide how best to introduce new tech, eval tech in use\\
        \textbf{Data Presentation}: aim to show how the prod being appropriated and integrated into surroundings, ex: vignettes (brief evocative description), excerpts, critical incidents, patterns, narratives\\
        \underline{\textbf{Gathering Data Key Issues}}: setting goals (how to analyze), identify participants (who to get data from), relationship with participants (clear \& professional, informed consent when appropriate), triangulation (look at data from more than 1 perspective, collect more than one type of data), pilot studies (small trial of main)\\
        \textbf{Data Recording}: notes, audio, video, photo can be used individually or in combination (notes + photo, audio + photo, video)\\
        \textbf{Interviews}: unstructured (no script, rich but unreplicable), structured (tightly scripted, like questionnaire), semi-structured (guided by script bit interesting issues can be explored in more depth, can provide balance between richness and replicability), focus groups (group interview)\\
        \textbf{Interview Qs}: closed (predetermined answer format, easier to analyze, may be analyzed by computer) vs open (no predetermined format), avoid long qs, compound sentences (split them), jargon \& lang that interviewee may not understand, leading qs that make assumptions (why do you like), unconscious bias (gender stereotype)\\
        \textbf{Running Interview}: intro (introduce self, explain goals, assure about ethical, ask to record, give informed consent form), warmup (first qs easy and non threatening), main body (present qs in logical order), cooloff (include few easy qs to defuse tension), closure (thank, signal end, turn recorder off)\\
        \textbf{Enriching Interview Proc}: props (devices for prompting interviewee, use proto, scenario)\\
        \textbf{Questionnaires}: can be administered to large populations, disseminated by paper, email, web\\
        \textbf{Questionnaire Design}: impact of q can be influenced by order, may need diff ver of questionnaire for diff pops, provide clear instruction on how to complete, strike balance between using white space and keeping questionnaire compact, avoid very long questionnaires, decide whether phrases will be all positive, negative or mixed\\
        \textbf{Q Format}: yes/no, checkboxes with many opts, rating scales, open ended responses\\
        \textbf{Encouraging good response}: make sure purpose of study is clear, promise anonymity, ensure questionnaire well designed, offer short version, if mailed include stamped addressed envelope, follow up with emails, phone calls, letters, provide incentive, 40\% response good, 20\% acceptable\\
        \textbf{Online Questionnaire Pros}: realtively easy \& quick to distribute, responses usually received quickly, no copying and postage costs, data can be collected in db for analysis, time required for data analysis is reduced, errors can be corrected easily\\
        \textbf{Online Questionnaire Cons}: sampling problematic if pop size unknown, preventing individuals from responding more than once can be a problem, also known to change questions in email questionnaires, server privacy laws maybe different\\
        \textbf{Observation}: direct observation in field (structuring frameworks, degree of participation (insider ot outsider), ethnography), direct observation in controlled environment, indirect observation (tracking users' activities, diaries, interaction logging, video \& photo collected remotely by drones or other equipment)\\
        \textbf{Observation Framework (Robson, 2014)}: space (what physical space is like, layout), actors, activities, objects, acts, events, time, goals, feelings\\
        \textbf{Planning \& Conducting Observation in Field}: decide on involvement (passive oberver to active participant), how to gain acceptance, how to handle sensitive topics (culture, private spaces), how to collect data (what data, what equipment, when to stop)\\
        \textbf{Ethnography}: philosophy with set of techniques that include participant observation and interviews, ethnographers immerse themselves in culture they study, degree of participation can vary from inside to outside, alanyzing vid \& data logs can be time consuming, collection of comments, incidents, artefacts made, coop of people beting observed is required, informants useful, data analysis continuous, qs get refined as understanding grows, reports usually contain examples\\
        \textbf{Online Ethnography}: virtual worlds have persistence that physical worlds do not have, ethical considerations \& presentation of results different\\
        \textbf{Ethnography Observations \& Materials}: activity/job descriptions, rules \& procedures that givern particular activities, descriptions of activities observed, recordings of talk between parties, informal interviews with participants explaining detail of observed activities, diagrams of physical layout, photos/videos/desctiptions of artefacts, workflow diagrams showing seq order of tasks, proc maps showing conn between acts\\
        \textbf{Observation in Controlled Env}: direct (think aloud), indirect (track user activities, diaries, interaction log, web analytics), both types use video, audio, photo, notes\\
        \textbf{Web Analytics}: sys of tools and techniques for optimizing web usage by measuring, collecting, analsing, reporting web data, tytpically focus on n of web visitors and page views\\
        \textbf{Choosing Techniques}: dep on study focus, participants involved, nature of techniques, resources \& time available\\
    \end{multicols}
    \end{document}
