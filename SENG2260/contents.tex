\documentclass[a4paper]{article}
    \usepackage[margin=0.25in]{geometry}
    \usepackage{multicol}

    \begin{document}
    \scriptsize
    \begin{multicols}{2}
        \noindent\textbf{HCI According to ACM}: the discipline concerned with design, evaluation, implementation of interactive computer systems for human use and with study of major phenomena surrounding them\\
        \textbf{HCI Definition}: study of interaction between people and computer based systems, concerned with physical, physiological and theoretical aspects of this process, about designing computer systems that support people so that they can carry out their activities productively and safely\\
        \textbf{User}: individual user, group of users working together or a sequence of users in organization dealing with some part of process/task\\
        \textbf{Computer}: tech ranging from desktop to large scale systems, or control/embedded systems\\
        \textbf{Interaction}: communication between user and computer in direct/indirect manner\\
        \textbf{What is involved}: study of humans using interfaces, development of new apps/systems to support user's activities, new devices and tools for users, develop usable products (easy to learn, effective to use, provide enjoyable and satisfying experience)\\
        \textbf{Interdisciplinary (HCI)}: computer science and system design are central concerns, but not possible to design effective interactive systems from one discipline in isolation\\
        \textbf{Contributing Disciplines}: cognitive psychology, computer science, anthropology, engineering, ergonomics and human factors, design, social and organizational psychology, sociology, philosophy, AI, linguistics\\
        \textbf{Importance}: how to make systems usable, evaluate usability of bespoke and COTS systems, understanding how users interact with computers and enabling users to do so effectively, matter of law (is suitable to task, easy to use and adaptable to user's knowledge and experience, provides feedback on performance, displays info in format and pace adapted to user, conforms to principles of software ergonomics)\\
        \textbf{Factors in HCI}: organisational, environmental, health and safety, the user, comfort, UI, task factors, constraints, system functionality, productivity factors, more:\\
        \textbf{Use and Context}: social organization and work, app areas, human-machine fit and adaptation\\
        \textbf{Human}: human information processing, language, communication and interaction, ergonomics\\
        \textbf{Computer}: I/O devices, dialogue techniques, dialogue genre, computer graphics, dialogue architecture\\
        \textbf{Development Process}: design approaches, implementation techniques and tools, evaluation techniques, example systems and case studies\\
        \textbf{Problems with Software}: excessive and unwanted share dealing in stock market, error in dosage given to patients receiving radiation therapy, erratic behaviour of military and civil aircraft, difficulties in controlling nuclear power plants during system failures, delays in dispatching ambulances to accidents\\
        \textbf{Avoiding Problematic Design + what to design}: take into account who users are, what activities are being carried out, where interaction is taking place, optimise interactions to they match users' activities and needs\\
        \textbf{3U}: Useful (accomplish what is required), Usable (do it easily and naturally, without danger of error), Used (make people want to use it, be attractive, engaging, fun)\\
        \textbf{Principles for supporting HCI + understanding users' needs}: take into account what people are good and bad at, consider what might help people with the way they currently do things, think through what might provide quality user experience, listening to what people want and getting them involved in design, listen to what people want and getting them involved in design, using tried and tested user based techniques during design process\\
        \textbf{Science or Craft}: bit of both (artistically pleasing and capable of fulfilling tasks required), innovative ideas lead to more usable systems (understand not only that they work but how and why they work), creative flow underpinned with science, scientific method accelerated by artistic insight\\
        \underline{\textbf{Bad Design}}: button and label look the same, buttons on different sides laid out different, need to push button first to activate instead of inserting bill (against convention)\\
        \textbf{Good Design}: marble answering machine (based on how everyday objects behave, easy, intuitive and pleasure to use, one step actions to perform core tasks), TiVo remote (peanut shape to fit in hand, logical layout, color coded distinctive buttons, easy to locate buttons)\\
        \textbf{Interaction Design (ID)}: designing interactive products to support the way people communicate and interact in their everyday and working lives (Preece, Sharp, Rogers, 2015), design of spaces for human communication and interaction (Winograd, 1997)\\
        \textbf{ID Goals}: develop usable products (easy to learn, effective to use and provide an enjoyable experience), involve users in design process\\
        \textbf{Interdisciplinary Contributor (ID)}: academic (phychology, social sciences, computing, engineering, ergonomics, informatics), design (graphic, product, artist, industrial, film industry)\\
        \textbf{Interdisciplinary Fields Doing ID}: HCI, ubiquitous computing, human factors, cognitive engineering, cognitive ergonomics, computer supported cooperative work, information systems\\
        \textbf{Working in Multidisciplinary Teams}: many people from different backgrounds involved, different prespectives and ways of seeing and talking about things, more ideas and designs generated, but difficult to communicate and progress forward the designs being created\\
        \textbf{ID in Business}: help companies enter age of consumer, design human centered products and services, from research and product to goal related design, provides wide range of design services, in each case targeted to address product development nneeds at hand, creates products, services, environmments for companies pioneering new ways to provide value to customers\\
        \textbf{Professionals in ID}: interaction designers (design of interactive aspects), usability engineers (evaluate products using usability methods and principles), web designers (develop and create visual design of websites, such as layouts), information architects (people who come up with ideas of how to plan and structure interactive products), UX designers (all before and field studies to inform product design)\\
        \textbf{UX}: how product behaves and is used by people in real world, way people feel about it and their pleasure and satisfaction when using it, looking at it, holding it, opening/closing it, cannot design UX, only design for UX\\
        \textbf{iPod}: quality UX from start, simple, elegant, distinct brand, pleasurable, must have fashion item, catchy names, cool\\
        \textbf{ID Process}: establishing requirements, developing alternatives, prototyping, evaluating\\
        \textbf{ID Characteristics}: users should be involded through development of the project, specific usability and user experience goals need to be identified, clearly documented and agreed at the beginning of the project, iteration neede through core activities\\
        \textbf{Help Designers}: understand how to design interactive products that fit with what people want, need desire, appreciate that one size does not fir all, identify incorrect assumptions about particular user groups (not all old people want/need big fonts), be aware of both people's sensitivities and capabilities\\
        \textbf{Cultural Differences}: date format\\
        \textbf{Accessibility}: degree to which product usable and accessible by as many people as possible, focus on disability (mental or physical impairment, adverse effect on everyday life, long term)\\
        \textbf{Usability Goals}: effective to use, efficient to use safe to use, have good utility, easy to learn, easy to remember how to use\\
        \textbf{Design Principles}: visibility (of action, invisible auto controls can be more difficult to use), feedback (sending information back to user), constraints (help prevent user from selecting incorrect options), consistency (easier to learn and use, but can break down, ex: control + first letter but problem if multiple commands with same first letter, internal is in app, external is across apps and devices), affordances (attribute of object that allows people to know how to use it, virtual interface better conceptualized as perceived affordances, learned conventions of arbitrary mappings between action and effect)\\
        \underline{\textbf{Understanding Problem Space}}: what you want to create, assumptions, will it achieve what you hope it will\\
        \textbf{Assumption}: taking something for granted when it needs further investigation (bad: people want to watch TV while driving, would not mind payng a lot more for 3D TV, ok: would not mind wearing 3D glasses in living rooms, enjoy enhanced clarity and color detail from 3D)\\
        \textbf{Claim}: state something to be true when it is still open to question (voice commands for GPS is safe)\\
        \textbf{Framework for analysing problem space}: are there problems with existing product/UX, why there are problems, how do you think proprosed design ideas might overcome these, if designing for new UX how proposed design ideas support, change, extend current ways of doing things\\
        \textbf{Benefits of Conceptualising}: orientation (enable design teams to ask specific questions about how conceptual model will be understood), open minded (prevent design teams from becoming narrowly focused early on), common ground (allow design teams to establish set of commonly agreed terms)\\
        \textbf{From Problem to Design Space}: understand problem can help inform design (what kind of interface, behaviour, functionality to provide), important to develop CM before\\
        \textbf{Conceptual Model (CM)}: high level description of how system organized and operates, enables designers to straighten out thinking before they start laying out their widgets (Johnson, Henderson, 2002), describe in terms of core activities and objects, also in terms of interface metaphors\\
        \textbf{CM components}: metaphors and analogies (understand what product is for and how to use it for activity), concpets people are exposed to through product (task-domain objects, attributes, ops: saving, revisiting, organizing), relationship and mappings between concepts\\
        \textbf{First steps in formulating CM}: what will users be doing when carrying out tasks, how will system support tasks, what kind of interface metaphor (if any) will be appropriate, what kinds of interaction modes and styles to use, always keep in mind when making design decisions how user will understand underlying conceptual model\\
        \textbf{Interface Metaphors}: conceptualizing what we are doing (surf web), conceptual model instantiated at the interface (desktop metaphor), visualizing op (icon of shopping cart for placing items into), designed to be similat to physical entity but also has own properties, can be based on activity, object or both, exploit user's familiar knowledge to help then understand unfamiliar, conjures up essence of unfamiliar activity, enabling users to leverage thos to understand more aspects of unfamiliar functionality ex: material (card, familiar form factor, material properties added, giving appearance and physical behaviour, like surface of paper)\\
        \textbf{Interface Metaphor Benefits}: makes learning new systems easier, helps users understand underlying conceptual model, can be very innovative and enable realm of computers and apps to be made more accessible to greater diversity of users\\
        \textbf{Interface Metaphor Problems}: break conventional and cultural rules (recycle bin on desktop), can constrain designers in way they conceptualize problem space, conflict with design principles, forces users to only understand system in terms of meethphor, designers can inadvertently use bad existing designs and transfer bad parts over, limit designers' imagination with coming up with new conceptual models\\
        \textbf{CM Interaction Types}: hybrid often used, support different ways to do same thing, can take longer to learn, details below
        \textbf{Instructing}: issue command, select option, quick and efficient interaction, good for repetitive kinds of actions performed on multiple objects, ex: word processor, vending machine\\
        \textbf{Conversing}: underlying model of conversation with human, range from simple voice recognition menu driven systems to more complex natural language dialogs, virtual agents, toys, robots designed to converse, ex: timetables, search engines, advice giving systems, help systems, allows users, especially novice and technophobes, to interact with system in way that is familiar (make them feel comfortable, at ease, less scared), but misunderstanding can arise when system does not know how to parse what user says\\
        \textbf{Manipulating}: involves drag, select, open, zoom on virtual objects, exploit user's knowledge of how they mode and manipulate in physical world, can involve actions useing physical controllers (Wii) or air gestures (Kinect) to control movements of on screen avatar, tagged physical objects that are maniplulated in physical world result in physical/digital events (animation)\\
        \textbf{Direct Manipulation (DM)}: continuous representation of objects and actions of interest, physical actions and button pressing instead of issuing commands with complex syntax, rapid reversible actions with immediate feedback on object of interest\\
        \textbf{DM Advantages}: novice can learn basic functionality quickly, experienced users can work extremely rapidly to carry out wide range of tasks (even defining new functions), intermittent users can retain operational concepts over time, error messages rarely needed, users can immediately see if actions are furthering goals and if not do something else, users experience less anxiety, users gain confidence and mastery and feel in control\\
        \textbf{DM Disadvantages}: some people take metaphor too literally, not all tasks can be described by objects and not all actions can be done directly, some tasks better achieved through delegating (spell checking), can become screen space gobblers, moving mouse can be slower than pressing function keys to do same actions\\
        \textbf{Exploring}: involves users moving through virtual or phyiscal environments (with embedded sensor tech)\\
        \textbf{CM Interface Types}: kind of interface used to support mode, ex: command, speech, data entry, form fill in, query, graphical, web, pen, VR/AR/Mixed, gesture, brain, when choosing need to determine requirements and user needs, take budget and other constraints into account, also depend on suitability of tech for activity being supported\\
        \textbf{Paradigm}: inspiration for CM, general approach adopted by community for carrying out research (shared assumptions, concepts, values, practices), ex: ubiquitous computing, pervasive computing, wearable computing, tangible bits, AR, attentive environments, ambient computing\\
        \textbf{Visions}: driving force that frames R\&D, invites people to imagine what life will be like in 10, 15, 20 years time (Apple 1987 Knowledge Navigator, Smart Cities, Smart health), provide concrete scenarios of how society can use next gen of imagined tech, also  raise questions concerning privacy and trust\\
        \textbf{Theory}: explanation of phenomenon (info processing that explains how the mind or some aspect of it is assumed to work), can help identify factors (cognitive, social, affective, relevant to design and evaluation of interactive products)\\
        \textbf{Models}: simplification of HCI phenomenon, intended to make it easier for designers to predict and evaluate alternative designs, abstracted from theory from contributing discipline (psychology, keystroke model)\\
        \textbf{Framework}: set of interrelated concepts and/or specific questions for `what to look for', provide advice on how to design\\
        \underline{\textbf{Why Need to Understand Users}}: interacting with tech is cognitive (need to take into account cognitive processes involved and cognitive limitations of users, provides knowledge about what users can and cannot be expected to do, identifies and explains nature and causes of problems users encounter, supply theories, modelling tools, guidance and methods that can lead to design of better interactive products)\\
        \textbf{Cognitive Processes}: below\\
        \textbf{Attention}: select things to concentrate on at point in time from mass of stimuli around, allow to focus on info relevant to what we are doing but limits ability to keep track of all events, involves audio and/or visual senses, info at interface should be structured to capture attention (use perceptual boundaries like windows, colour, sound, ordering, spacing, underlining, sequencing, animation), multitaskers easily distracted and find it hard to filter irrelevant info\\
        \textbf{Design Implications for Attention}: make info noticeable when it needs attending to, use tachniques that make things stand ou, avoid clutter interface with too much information (search engine and form fill in that have simple and clean interface are easier to use)\\
        \textbf{Perception}: how info acquired from world and transformed to experiences, text should be legible, icons should be easy to distinguish and read, group information\\
        \textbf{Design Implications for Perception}: icons should enable users to readily distinguish meaning, bordering and spacing are effective visual ways of grouping info, sounds should be audible and distinguishable, speech output should enable users to distinguish between set of spoken words, text should be legible and distinguishable from background, tactile feedback should allow users to recognize and distinguish different meanings\\
        \textbf{Memory}: involves first encoding (more attention paid, more it is processed in terms of thinking and comparing with other knowledge, more likely to be remembered), then retrieving knowledge, don't remember everything (involves filtering and processing what is attended to), context important in affecting memory (where, when), recognize better than recall, remember less about photographed objects than actually seen (Henkel, 2014), people can scan list to find one they want\\
        \textbf{Digital Content Management}: memory involves 2 processes (recall directed and recognition based scanning), file management syystems should be designed to optimize both (search box and history), help users encode files in richer ways (colour, flagging, image, flexible text, timestamp)\\
        \textbf{SenseCam}: intermittently takes photos without user intervention while worn, can improve memory from Alzheimer's\\
        \textbf{Design Implications}: don;t overload memory with complicated procedures, recognition vs recall, provide various ways of encoding info\\
        \textbf{Learning}: prefer to learn by doing rather than read manual, rely more on internet to look things up, expecting to have internet reduces need and extent to which we remember, enhances memory of knowing where to find it online (Sparrow et al, 2011)\\
        \textbf{Design Implications}: design interfaces that encourage exploration, constrain and guide learners, dynamically linking concepts and representations can facilitate learning of complex material\\
        \textbf{Reading, Speaking, Listening}: many prefer listening to reading, reading can be quicker than others, listening requires less cognitive effort than others, dyslexics have difficulties understanding and recognizing written words, speech recognition, speech output, natural language systems (type in questions and give text based responses)\\
        \textbf {Design Implications}: speech based menus and instructions should be short, accentuate intonation of artificial voice, provide opportunities for making text large on screen\\
        \textbf{Problem Solving, Planning, Reasoning, Decision Making}: involves reflective cognition (thinking about what to do, what the options are, consequences), often involves conscious processes, discussion with others or self, use of artefacts (maps, books, pen, paper), may involve working though different scenarios and deciding which is best option\\
        \textbf{Design Implications}: provide additional info/functions for users who wish to understand more about how to carry out an activity more effectively, use simple computational aids to support rapid decision making and planning for users on the move\\
        \textbf{App Mentality}: developing in psyche of younger generation is making it worse for them to make their own decisions because they are becoming risk averse (Gardner, Davis, 2013), all desires and questions should be satisfied/answered by app, so less thinking?\\
        \textbf{Mental Model}: how to use the system (what to do next), what to do with unfamiliar systems or unexpected situations (how system works), used to make inferences, involves unconscious and conscious processes (images and analogies activated), deep (how to drive car) vs shallow (how car works), errorneous ex: turn up thermostat to heat up quicker\\
        \textbf{Gulf of Execution and Evaluation}: exec is from user to sys, eval is other way\\
        \textbf{Info Processing Steps}: encoding, comparison, response select, exec\\
        \textbf{Distributed Cognition (DC)}: info transformed through different media (computer, display, paper, head) instead of in mind only\\
        \textbf{DC Involves}: distributed problem solving, role of verbal and on verbal behaviour, various coordinating mechanisms used (rules, procedures), comms that takes place as collaborative activity progresses, how knowledge shared and accessed\\
        \textbf{External Cognition}: concerned with explaining how we interact with external representations (maps, notes, diagrams), what are cognitive benefits, what processes involved, how they extend cognition, what computer based representations can we develop to help even more\\
        \textbf{Externalizing to reduce memory load}: remind that we need to do something, remind what to do, remind when to do, ex: diaries, reminders, calendars, notes, shopping \& todo lists, post-its, piles, marked emails\\
        \textbf{Computational Offloading}: when tool is used in conjunction with external representation to carry out computation (pen and paper)\\
        \textbf{Annotation}: modify existing representations through making marks (cross off, tick, underline)\\
        \textbf{Cognitive Tracing}: externally manipulating items into different orders/structures (playing Scrabble, cards)\\
        \textbf{Design Implication}: provide external representations at interface that memory load and facilitate computational offloading (info visualizations to allow people to make sense of and make rapid decisions about big data)\\
    \end{multicols}
    \end{document}
