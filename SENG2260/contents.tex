\documentclass[a4paper]{article}
    \usepackage[margin=0.25in]{geometry}
    \usepackage{multicol}

    \begin{document}
    \scriptsize
    \begin{multicols}{2}
        \noindent\textbf{HCI According to ACM}: the discipline concerned with design, evaluation, implementation of interactive computer systems for human use and with study of major phenomena surrounding them\\
        \textbf{HCI Definition}: study of interaction between people and computer bases systems, concerned with physical, physiological and theoretical aspects of this process, about designing computer systems that support people so that they can carry out their activities productively and safely\\
        \textbf{User}: individual user, group of users working together or a sequence of users in organization dealing with some part of process/task\\
        \textbf{Computer}: tech ranging from desktop to large scale systems, or control/embedded systems\\
        \textbf{Interaction}: communication between user and computer in direct/indirect manner\\
        \textbf{What is involved}: study of humans using interfaces, development of new apps/systems to support user's activities, new devices and tools for users, develop usable products (easy to learn, effective to use, provide enjoyable and satisfying experience)\\
        \textbf{Interdisciplinary}: computer science and system design are central concerns, but not possible to design effective interactive systems from one discipline in isolation\\
        \textbf{Contributing Disciplines}: cognitive psychology, computer science, anthropology, engineering, ergonomics and human factors, design, social and organizational psychology, sociology, philosophy, AI, linguistics\\
        \textbf{Importance}: how to make systems usable, evaluate usability of bespoke and COTS systems, understanding how users interact with computers and enabling users to do so effectively, matter of law (is suitable to task, easy to use and adaptable to user's knowledge and experience, provides feedback on performance, displays info in format and pace adapted to user, conforms to principles of software ergonomics)\\
        \textbf{Factors in HCI}: organisational, environmental, health and safety, the user, comfort, UI, task factors, constraints, system functionality, productivity factors, more:\\
        \textbf{Use and Context}: social organization and work, app areas, human-machine fit and adaptation\\
        \textbf{Human}: human information processing, language, communication and interaction, ergonomics\\
        \textbf{Computer}: I/O devices, dialogue techniques, dialogue genre, computer graphics, dialogue architecture\\
        \textbf{Development Process}: design approaches, implementation techniques and tools, evaluation techniques, example systems and case studies\\
        \textbf{Problems with Software}: excessive and unwanted share dealing in stock market, error in dosage given to patients receiving radiation therapy, erratic behaviour of military and civil aircraft, difficulties in controlling nuclear power plants during system failures, delays in dispatching ambulances to accidents\\
        \textbf{Avoiding Problematic Design}: take into account who users are, what activities are being carried out, where interaction is taking place, optimise interactions to they match users' activities and needs\\
        \textbf{3U}: Useful (accomplish what is required), Usable (do it easily and naturally, without danger of error), Used (make people want to use it, be attractive, engaging, fun)\\
        \textbf{Principles for supporting HCI}: take into account what people are good and bad at, consider what might help people with the way they currently do things, think through what might provide quality user experience, listening to what people want and getting them involved in design, listen to what people want and getting them involved in design, using tried and tested user based techniques during design process\\
        \textbf{Science or Craft}: bit of both (artistically pleasing and capable of fulfilling tasks required), innovative ideas lead to more usable systems (understand not only that they work but how and why they work), creative flow underpinned with science, scientific method accelerated by artistic insight\\
    \end{multicols}
    \end{document}
