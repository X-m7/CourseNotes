\documentclass[a4paper]{article}
    \usepackage[margin=0.25in]{geometry}
    \usepackage{multicol}

    \begin{document}
    \scriptsize
    \begin{multicols}{2}
        \noindent\textbf{HCI According to ACM}: the discipline concerned with design, evaluation, implementation of interactive computer systems for human use and with study of major phenomena surrounding them\\
        \textbf{HCI Definition}: study of interaction between people and computer bases systems, concerned with physical, physiological and theoretical aspects of this process, about designing computer systems that support people so that they can carry out their activities productively and safely\\
        \textbf{User}: individual user, group of users working together or a sequence of users in organization dealing with some part of process/task\\
        \textbf{Computer}: tech ranging from desktop to large scale systems, or control/embedded systems\\
        \textbf{Interaction}: communication between user and computer in direct/indirect manner\\
        \textbf{What is involved}: study of humans using interfaces, development of new apps/systems to support user's activities, new devices and tools for users, develop usable products (easy to learn, effective to use, provide enjoyable and satisfying experience)\\
        \textbf{Interdisciplinary (HCI)}: computer science and system design are central concerns, but not possible to design effective interactive systems from one discipline in isolation\\
        \textbf{Contributing Disciplines}: cognitive psychology, computer science, anthropology, engineering, ergonomics and human factors, design, social and organizational psychology, sociology, philosophy, AI, linguistics\\
        \textbf{Importance}: how to make systems usable, evaluate usability of bespoke and COTS systems, understanding how users interact with computers and enabling users to do so effectively, matter of law (is suitable to task, easy to use and adaptable to user's knowledge and experience, provides feedback on performance, displays info in format and pace adapted to user, conforms to principles of software ergonomics)\\
        \textbf{Factors in HCI}: organisational, environmental, health and safety, the user, comfort, UI, task factors, constraints, system functionality, productivity factors, more:\\
        \textbf{Use and Context}: social organization and work, app areas, human-machine fit and adaptation\\
        \textbf{Human}: human information processing, language, communication and interaction, ergonomics\\
        \textbf{Computer}: I/O devices, dialogue techniques, dialogue genre, computer graphics, dialogue architecture\\
        \textbf{Development Process}: design approaches, implementation techniques and tools, evaluation techniques, example systems and case studies\\
        \textbf{Problems with Software}: excessive and unwanted share dealing in stock market, error in dosage given to patients receiving radiation therapy, erratic behaviour of military and civil aircraft, difficulties in controlling nuclear power plants during system failures, delays in dispatching ambulances to accidents\\
        \textbf{Avoiding Problematic Design + what to design}: take into account who users are, what activities are being carried out, where interaction is taking place, optimise interactions to they match users' activities and needs\\
        \textbf{3U}: Useful (accomplish what is required), Usable (do it easily and naturally, without danger of error), Used (make people want to use it, be attractive, engaging, fun)\\
        \textbf{Principles for supporting HCI + understanding users' needs}: take into account what people are good and bad at, consider what might help people with the way they currently do things, think through what might provide quality user experience, listening to what people want and getting them involved in design, listen to what people want and getting them involved in design, using tried and tested user based techniques during design process\\
        \textbf{Science or Craft}: bit of both (artistically pleasing and capable of fulfilling tasks required), innovative ideas lead to more usable systems (understand not only that they work but how and why they work), creative flow underpinned with science, scientific method accelerated by artistic insight\\
        \underline{\textbf{Bad Design}}: button and label look the same, buttons on different sides laid out different, need to push button first to activate instead of inserting bill (against convention)\\
        \textbf{Good Design}: marble answering machine (based on how everyday objects behave, easy, intuitive and pleasure to use, one step actions to perform core tasks), TiVo remote (peanut shape to fit in hand, logical layout, color coded distinctive buttons, easy to locate buttons)\\
        \textbf{Interaction Design (ID)}: designing interactive products to support the way people communicate and interact in their everyday and working lives (Preece, Sharp, Rogers, 2015), design of spaces for human communication and interaction (Winograd, 1997)\\
        \textbf{ID Goals}: develop usable products (easy to learn, effective to use and provide an enjoyable experience), involve users in design process\\
        \textbf{Interdisciplinary Contributor (ID)}: academic (phychology, social sciences, computing, engineering, ergonomics, informatics), design (graphic, product, artist, industrial, film industry)\\
        \textbf{Interdisciplinary Fields Doing ID}: HCI, ubiquitous computing, human factors, cognitive engineering, cognitive ergonomics, computer supported cooperative work, information systems\\
        \textbf{Working in Multidisciplinary Teams}: many people from different backgrounds involved, different prespectives and ways of seeing and talking about things, more ideas and designs generated, but difficult to communicate and progress forward the designs being created\\
        \textbf{ID in Business}: help companies enter age of consumer, design human centered products and services, from research and product to goal related design, provides wide range of design services, in each case targeted to address product development nneeds at hand, creates products, services, environmments for companies pioneering new ways to provide value to customers\\
        \textbf{Professionals in ID}: interaction designers (design of interactive aspects), usability engineers (evaluate products using usability methods and principles), web designers (develop and create visual design of websites, such as layouts), information architects (people who come up with ideas of how to plan and structure interactive products), UX designers (all before and field studies to inform product design)\\
        \textbf{UX}: how product behaves and is used by people in real world, way people feel about it and their pleasure and satisfaction when using it, looking at it, holding it, opening/closing it, cannot design UX, only design for UX\\
        \textbf{iPod}: quality UX from start, simple, elegant, distinct brand, pleasurable, must have fashion item, catchy names, cool\\
        \textbf{ID Process}: establishing requirements, developing alternatives, prototyping, evaluating\\
        \textbf{ID Characteristics}: users should be involded through development of the project, specific usability and user experience goals need to be identified, clearly documented and agreed at the beginning of the project, iteration neede through core activities\\
        \textbf{Help Designers}: understand how to design interactive products that fit with what people want, need desire, appreciate that one size does not fir all, identify incorrect assumptions about particular user groups (not all old people want/need big fonts), be aware of both people's sensitivities and capabilities\\
        \textbf{Cultural Differences}: date format\\
        \textbf{Accessibility}: degree to which product usable and accessible by as many people as possible, focus on disability (mental or physical impairment, adverse effect on everyday life, long term)\\
        \textbf{Usability Goals}: effective to use, efficient to use safe to use, have good utility, easy to learn, easy to remember how to use\\
        \textbf{Design Principles}: visibility (of action, invisible auto controls can be more difficult to use), feedback (sending information back to user), constraints (help prevent user from selecting incorrect options), consistency (easier to learn and use, but can break down, ex: control + first letter but problem if multiple commands with same first letter, internal is in app, external is across apps and devices), affordances (attribute of object that allows people to know how to use it, virtual interface better conceptualized as perceived affordances, learned conventions of arbitrary mappings between action and effect)\\
    \end{multicols}
    \end{document}
